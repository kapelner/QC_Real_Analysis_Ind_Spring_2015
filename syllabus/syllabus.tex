\documentclass[12pt]{article}

\usepackage{datetime}
\usepackage[auth-sc,affil-sl]{authblk}
\usepackage{color}
\usepackage{placeins}
\usepackage{enumerate}

\include{preamble}

\newcommand{\coursewebpage}{\href{https://github.com/kapelner/QC_Real_Analysis_Ind_Spring_2015}{course homepage}}


\title{MATH 6xx Spring 2015 (x credits) \\ Course Syllabus}

\author[]{Adam Kapelner, Ph.D.}

\affil[]{Queens College, City University of New York}
\settimeformat{ampmtime}
\date{\small document last updated \today ~\currenttime }

\begin{document}
\maketitle

\begin{table}[htp]
\centering
\begin{tabular}{rl}
Instructor & Professor Adam Kapelner \\
Office & 325 Kissena Hall I (64-19 Kissena Blvd between 64 \& 65 Ave) \\
Contact & \url{kapelner@qc.cuny.edu} \\
Lecture Time / Loc & TBA, Kissena Hall 325 \\
Course Homepage & \coursewebpage
\end{tabular}
\end{table}

\section*{Course Overview}

Math 6xx is a course in real analysis from the ground up. We will be using 

\begin{itemize}
\itemsep -0.0em 
\item Set theory and cardinality theory
\item Ordered sets and fields
\item Construction of the sets of $\naturals$, $\rationals$ and $\reals$
\item Basic Topology
\item Sequences and Series
\item Continuity
\item Limts
\item Differentiation
\end{itemize}

This is not your typical mathematics course. This course will focus on mathematical proofs and \textit{not} computation. This course will teach you how to think.


\section*{Course Materials}

\paragraph{Textbook:} Principles of Mathematical Analysis 3rd edition by Walter Rudin. You can buy this used on \href{http://www.abebooks.com/servlet/SearchResults?kn=rudin&tn=Principles+of+Mathematical+Analysis}{the Internet} for less than \$10.

\paragraph{Computer Software:} You will be required to write your homeworks in \LaTeX. I recommend using \href{http://overleaf.com}{overleaf} to write up your homeworks. This has the advantage of (a) not having to install anything on your computer and not having to maintain your \LaTeX ~installation (b) allowing easy collaboration with others (c) always having a backup of your work since it's always on the cloud. If you insist to have \LaTeX ~running on your computer, you can download it for Windows \href{http://www.miktex.org/download}{here} and for MAC \href{http://www.tug.org/mactex/}{here}. For editing and producing PDF's, I recommend \TeX works which can be downloaded \href{http://www.tug.org/texworks/#Getting_TeXworks}{here}. 

\paragraph{Calculator:} None needed.

\section*{Announcements}

Announcements will be made via email. I am known to send a couple emails per week on important issues.

\section*{Lectures}

I have a no computer / tablet / phone policy during lectures. Only pen / pencil and paper. Classes are 60 minutes twice a week and run from Thursday, January 29 until Thursday, May 14. There will be 28 classes. The format of the classes will not be lectures exclusively. They will be mostly problem solving sessions as we move through the textbook.


\section*{Homework}

There will be 11--13 homework assignments. Homeworks will be assigned and placed on the \coursewebpage~ and will usually be due a week later in class. Homework will be \textbf{graded} out of 100 with extra credit getting scores possibly $> 100$. I will be doing the grading. 

Homework must be typeset in \LaTeX, printed, neat and stapled (\textbf{it cannot be emailed to me}). Homework can be given to me in class or delivered to my office in Kissena Hall. \textit{Homework cannot be handed in to my mail slot in the Kiely mathematics office} (unless you want it to be counted as late).

Graded homework will be returned in class. Regrades are handled right after class is over.

You are encouraged to seek help from the instructor if you have questions. \ingreen{You are recommended to work with each other and help each other.} You must, however, submit your own solutions, \textit{with your own write-up} and in \textit{your own words}. There can be no collaboration on the actual \textit{writing}. Failure to comply will result in severe penalties. The university honor code is something I take seriously and I send people to the Dean every semester for violations.

\subsection*{Philosophy of Homework}


Homework is the \textit{most} important part of this course. Success in analysis courses comes from experience in proof methods and it involves working with and thinking about the concepts. It's kind of like weightlifting; you have to lift weights to build muscles. My job as an instructor is to provide assistance through your \href{http://en.wikipedia.org/wiki/Zone_of_proximal_development}{zone of proximal development}. With me, you can grow more than you can alone. To this effect, homework problems are color coded \ingreen{green} for easy, \inyellow{yellow} for harder, \inred{red} for challenging and \inpurple{purple} for extra credit. You need to know how to do all the greens by yourself. If you've been to class and took notes, they are a joke. Yellows and reds feel free to work with others. Only do extra credits if you have already finished the assignment.

\subsection*{Time Spent on Homework }

This is a x credit course. Thus, the amount of work outside of the 2hr in-class time per week is x-x hours. I will aim for x hr of homework per week on average. However, doing the homework well is your sole responsibility since I will make sure that by doing the homework you will study and understand the concepts in the lectures.

\subsection*{Late Homework}

Late homework will be penalized 10 points per day for a maximum of five days. Do not ask for extensions; just hand in the homework late. After five days, it will receive a zero.


\section*{Examinations}

Examinations are TBA. I may opt for (a) no exams (b) a take-home midterm and final (c) one in-class final.

%\subsection*{Exam Materials}
%
%I allow you to bring any calculator you wish but it cannot be your phone. The only other items allowed are pencil and eraser. I do not recommend using pen but it is allowed
%
%I also allow \qu{cheat sheets} on examinations. For both midterms, you are allowed to bring one 8.5'' $\times$ 11'' sheet of paper (front and back). On this paper you can write anything you would like which you believe will help you on the exam. For the final, you are allowed to bring three 8.5'' $\times$ 11'' sheet of paper (front and back). I will be handing back the cheat sheets so you can reuse your midterm cheat sheets for the final if you wish. 


\subsection*{Cheating on Exams}

If I catch you cheating, you can either take a zero on the exam, or you can roll the dice before the University Honor Council which may choose to suspend you.


%\subsection*{Missing Exams}
%
%There are no make-up exams. If you miss the exam, you get a zero. If you are sick, I need documentation of your visit to a hospital or doctor. Expect my grader to call the doctor or hospital to verify the legitimacy of your note. If you need to leave the country for an emergency, I will expect proper documentation as well.



\section*{Grading and Grading Policy}

Your course grade will be calculated based on the percentages as follows: 

\begin{table}[h]
\centering
\begin{tabular}{l|l}
Homework & 70\% \\
Examinations & 25\%
\end{tabular}
\end{table}
\FloatBarrier


\section*{Auditing}

Auditors are welcome in both sections. They are encouraged to do all homework assignments. I will even grade them. Note that the university does not allow auditors to take examinations.

\end{document}
