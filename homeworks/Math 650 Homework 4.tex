\documentclass[12pt]{article} 
\usepackage{amsmath}
\usepackage{amsfonts}
\usepackage{amssymb}
\usepackage{color}

\newtheorem{theorem}{Theorem}[section]
\newtheorem{corollary}{Corollary}[theorem]
\newtheorem{lemma}[theorem]{Lemma}
\title{Math 650.2 Homework 4}
\author{Elliot Gangaram\\
\date{}
\ elliot.gangaram@gmail.com \\}
%packages
%\usepackage{latexsym}
\usepackage{graphicx}
\usepackage{color}
\usepackage{amsmath}
\usepackage{dsfont}
\usepackage{placeins}
\usepackage{amssymb}
\usepackage{wasysym}
\usepackage{abstract}
\usepackage{hyperref}
\usepackage{etoolbox}
\usepackage{datetime}
\usepackage{xcolor}
\usepackage{alphalph}
\settimeformat{ampmtime}

%\usepackage{pstricks,pst-node,pst-tree}

%\usepackage{algpseudocode}
%\usepackage{amsthm}
%\usepackage{hyperref}
%\usepackage{mathrsfs}
%\usepackage{amsfonts}
%\usepackage{bbding}
%\usepackage{listings}
%\usepackage{appendix}
\usepackage[margin=1in]{geometry}
%\geometry{papersize={8.5in,11in},total={6.5in,9in}}
%\usepackage{cancel}
%\usepackage{algorithmic, algorithm}

\makeatletter
\def\maxwidth{ %
  \ifdim\Gin@nat@width>\linewidth
    \linewidth
  \else
    \Gin@nat@width
  \fi
}
\makeatother

\definecolor{fgcolor}{rgb}{0.345, 0.345, 0.345}
\newcommand{\hlnum}[1]{\textcolor[rgb]{0.686,0.059,0.569}{#1}}%
\newcommand{\hlstr}[1]{\textcolor[rgb]{0.192,0.494,0.8}{#1}}%
\newcommand{\hlcom}[1]{\textcolor[rgb]{0.678,0.584,0.686}{\textit{#1}}}%
\newcommand{\hlopt}[1]{\textcolor[rgb]{0,0,0}{#1}}%
\newcommand{\hlstd}[1]{\textcolor[rgb]{0.345,0.345,0.345}{#1}}%
\newcommand{\hlkwa}[1]{\textcolor[rgb]{0.161,0.373,0.58}{\textbf{#1}}}%
\newcommand{\hlkwb}[1]{\textcolor[rgb]{0.69,0.353,0.396}{#1}}%
\newcommand{\hlkwc}[1]{\textcolor[rgb]{0.333,0.667,0.333}{#1}}%
\newcommand{\hlkwd}[1]{\textcolor[rgb]{0.737,0.353,0.396}{\textbf{#1}}}%

\usepackage{framed}
\makeatletter
\newenvironment{kframe}{%
 \def\at@end@of@kframe{}%
 \ifinner\ifhmode%
  \def\at@end@of@kframe{\end{minipage}}%
  \begin{minipage}{\columnwidth}%
 \fi\fi%
 \def\FrameCommand##1{\hskip\@totalleftmargin \hskip-\fboxsep
 \colorbox{shadecolor}{##1}\hskip-\fboxsep
     % There is no \\@totalrightmargin, so:
     \hskip-\linewidth \hskip-\@totalleftmargin \hskip\columnwidth}%
 \MakeFramed {\advance\hsize-\width
   \@totalleftmargin\z@ \linewidth\hsize
   \@setminipage}}%
 {\par\unskip\endMakeFramed%
 \at@end@of@kframe}
\makeatother

\definecolor{shadecolor}{rgb}{.77, .77, .77}
\definecolor{messagecolor}{rgb}{0, 0, 0}
\definecolor{warningcolor}{rgb}{1, 0, 1}
\definecolor{errorcolor}{rgb}{1, 0, 0}
\newenvironment{knitrout}{}{} % an empty environment to be redefined in TeX

\usepackage{alltt}
\usepackage[T1]{fontenc}

\newcommand{\qu}[1]{``#1''}
\newcounter{probnum}
\setcounter{probnum}{1}

%create definition to allow local margin changes
\def\changemargin#1#2{\list{}{\rightmargin#2\leftmargin#1}\item[]}
\let\endchangemargin=\endlist 

%allow equations to span multiple pages
\allowdisplaybreaks

%define colors and color typesetting conveniences
\definecolor{gray}{rgb}{0.5,0.5,0.5}
\definecolor{black}{rgb}{0,0,0}
\definecolor{white}{rgb}{1,1,1}
\definecolor{blue}{rgb}{0.5,0.5,1}
\newcommand{\inblue}[1]{\color{blue}#1 \color{black}}
\definecolor{green}{rgb}{0.133,0.545,0.133}
\newcommand{\ingreen}[1]{\color{green}#1 \color{black}}
\definecolor{yellow}{rgb}{1,1,0}
\newcommand{\inyellow}[1]{\color{yellow}#1 \color{black}}
\definecolor{orange}{rgb}{0.9,0.649,0}
\newcommand{\inorange}[1]{\color{orange}#1 \color{black}}
\definecolor{red}{rgb}{1,0.133,0.133}
\newcommand{\inred}[1]{\color{red}#1 \color{black}}
\definecolor{purple}{rgb}{0.58,0,0.827}
\newcommand{\inpurple}[1]{\color{purple}#1 \color{black}}
\definecolor{backgcode}{rgb}{0.97,0.97,0.8}
\definecolor{Brown}{cmyk}{0,0.81,1,0.60}
\definecolor{OliveGreen}{cmyk}{0.64,0,0.95,0.40}
\definecolor{CadetBlue}{cmyk}{0.62,0.57,0.23,0}

%define new math operators
\DeclareMathOperator*{\argmax}{arg\,max~}
\DeclareMathOperator*{\argmin}{arg\,min~}
\DeclareMathOperator*{\argsup}{arg\,sup~}
\DeclareMathOperator*{\arginf}{arg\,inf~}
\DeclareMathOperator*{\convolution}{\text{\Huge{$\ast$}}}
\newcommand{\infconv}[2]{\convolution^\infty_{#1 = 1} #2}
%true functions

%%%% GENERAL SHORTCUTS

%shortcuts for pure typesetting conveniences
\newcommand{\bv}[1]{\boldsymbol{#1}}

%shortcuts for compound constants
\newcommand{\BetaDistrConst}{\dfrac{\Gamma(\alpha + \beta)}{\Gamma(\alpha)\Gamma(\beta)}}
\newcommand{\NormDistrConst}{\dfrac{1}{\sqrt{2\pi\sigma^2}}}

%shortcuts for conventional symbols
\newcommand{\tsq}{\tau^2}
\newcommand{\tsqh}{\hat{\tau}^2}
\newcommand{\sigsq}{\sigma^2}
\newcommand{\sigsqsq}{\parens{\sigma^2}^2}
\newcommand{\sigsqovern}{\dfrac{\sigsq}{n}}
\newcommand{\tausq}{\tau^2}
\newcommand{\tausqalpha}{\tau^2_\alpha}
\newcommand{\tausqbeta}{\tau^2_\beta}
\newcommand{\tausqsigma}{\tau^2_\sigma}
\newcommand{\betasq}{\beta^2}
\newcommand{\sigsqvec}{\bv{\sigma}^2}
\newcommand{\sigsqhat}{\hat{\sigma}^2}
\newcommand{\sigsqhatmlebayes}{\sigsqhat_{\text{Bayes, MLE}}}
\newcommand{\sigsqhatmle}[1]{\sigsqhat_{#1, \text{MLE}}}
\newcommand{\bSigma}{\bv{\Sigma}}
\newcommand{\bSigmainv}{\bSigma^{-1}}
\newcommand{\thetavec}{\bv{\theta}}
\newcommand{\thetahat}{\hat{\theta}}
\newcommand{\thetahatmle}{\hat{\theta}_{\mathrm{MLE}}}
\newcommand{\thetavechatmle}{\hat{\thetavec}_{\mathrm{MLE}}}
\newcommand{\muhat}{\hat{\mu}}
\newcommand{\musq}{\mu^2}
\newcommand{\muvec}{\bv{\mu}}
\newcommand{\muhatmle}{\muhat_{\text{MLE}}}
\newcommand{\lambdahat}{\hat{\lambda}}
\newcommand{\lambdahatmle}{\lambdahat_{\text{MLE}}}
\newcommand{\etavec}{\bv{\eta}}
\newcommand{\alphavec}{\bv{\alpha}}
\newcommand{\minimaxdec}{\delta^*_{\mathrm{mm}}}
\newcommand{\ybar}{\bar{y}}
\newcommand{\xbar}{\bar{x}}
\newcommand{\Xbar}{\bar{X}}
\newcommand{\phat}{\hat{p}}
\newcommand{\Phat}{\hat{P}}
\newcommand{\Zbar}{\bar{Z}}
\newcommand{\iid}{~{\buildrel iid \over \sim}~}
\newcommand{\inddist}{~{\buildrel ind \over \sim}~}
\newcommand{\approxdist}{~{\buildrel approx \over \sim}~}
\newcommand{\equalsindist}{~{\buildrel d \over =}~}
\newcommand{\loglik}[1]{\ell\parens{#1}}
\newcommand{\thetahatkminone}{\thetahat^{(k-1)}}
\newcommand{\thetahatkplusone}{\thetahat^{(k+1)}}
\newcommand{\thetahatk}{\thetahat^{(k)}}
\newcommand{\half}{\frac{1}{2}}
\newcommand{\third}{\frac{1}{3}}
\newcommand{\twothirds}{\frac{2}{3}}
\newcommand{\fourth}{\frac{1}{4}}
\newcommand{\fifth}{\frac{1}{5}}
\newcommand{\sixth}{\frac{1}{6}}

%shortcuts for vector and matrix notation
\newcommand{\A}{\bv{A}}
\newcommand{\At}{\A^T}
\newcommand{\Ainv}{\inverse{\A}}
\newcommand{\B}{\bv{B}}
\newcommand{\K}{\bv{K}}
\newcommand{\Kt}{\K^T}
\newcommand{\Kinv}{\inverse{K}}
\newcommand{\Kinvt}{(\Kinv)^T}
\newcommand{\M}{\bv{M}}
\newcommand{\Bt}{\B^T}
\newcommand{\Q}{\bv{Q}}
\newcommand{\Qt}{\Q^T}
\newcommand{\R}{\bv{R}}
\newcommand{\Rt}{\R^T}
\newcommand{\Z}{\bv{Z}}
\newcommand{\X}{\bv{X}}
\newcommand{\Xsub}{\X_{\text{(sub)}}}
\newcommand{\Xsubadj}{\X_{\text{(sub,adj)}}}
\newcommand{\I}{\bv{I}}
\newcommand{\Y}{\bv{Y}}
\newcommand{\sigsqI}{\sigsq\I}
\renewcommand{\P}{\bv{P}}
\newcommand{\Psub}{\P_{\text{(sub)}}}
\newcommand{\Pt}{\P^T}
\newcommand{\Pii}{P_{ii}}
\newcommand{\Pij}{P_{ij}}
\newcommand{\IminP}{(\I-\P)}
\newcommand{\Xt}{\bv{X}^T}
\newcommand{\XtX}{\Xt\X}
\newcommand{\XtXinv}{\parens{\Xt\X}^{-1}}
\newcommand{\XtXinvXt}{\XtXinv\Xt}
\newcommand{\XXtXinvXt}{\X\XtXinvXt}
\newcommand{\x}{\bv{x}}
\newcommand{\onevec}{\bv{1}}
\newcommand{\oneton}{1, \ldots, n}
\newcommand{\yoneton}{y_1, \ldots, y_n}
\newcommand{\yonetonorder}{y_{(1)}, \ldots, y_{(n)}}
\newcommand{\Yoneton}{Y_1, \ldots, Y_n}
\newcommand{\iinoneton}{i \in \braces{\oneton}}
\newcommand{\onetom}{1, \ldots, m}
\newcommand{\jinonetom}{j \in \braces{\onetom}}
\newcommand{\xoneton}{x_1, \ldots, x_n}
\newcommand{\Xoneton}{X_1, \ldots, X_n}
\newcommand{\xt}{\x^T}
\newcommand{\y}{\bv{y}}
\newcommand{\yt}{\y^T}
\renewcommand{\c}{\bv{c}}
\newcommand{\ct}{\c^T}
\newcommand{\tstar}{\bv{t}^*}
\renewcommand{\u}{\bv{u}}
\renewcommand{\v}{\bv{v}}
\renewcommand{\a}{\bv{a}}
\newcommand{\s}{\bv{s}}
\newcommand{\yadj}{\y_{\text{(adj)}}}
\newcommand{\xjadj}{\x_{j\text{(adj)}}}
\newcommand{\xjadjM}{\x_{j \perp M}}
\newcommand{\yhat}{\hat{\y}}
\newcommand{\yhatsub}{\yhat_{\text{(sub)}}}
\newcommand{\yhatstar}{\yhat^*}
\newcommand{\yhatstarnew}{\yhatstar_{\text{new}}}
\newcommand{\z}{\bv{z}}
\newcommand{\zt}{\z^T}
\newcommand{\bb}{\bv{b}}
\newcommand{\bbt}{\bb^T}
\newcommand{\bbeta}{\bv{\beta}}
\newcommand{\beps}{\bv{\epsilon}}
\newcommand{\bepst}{\beps^T}
\newcommand{\e}{\bv{e}}
\newcommand{\Mofy}{\M(\y)}
\newcommand{\KofAlpha}{K(\alpha)}
\newcommand{\ellset}{\mathcal{L}}
\newcommand{\oneminalph}{1-\alpha}
\newcommand{\SSE}{\text{SSE}}
\newcommand{\SSEsub}{\text{SSE}_{\text{(sub)}}}
\newcommand{\MSE}{\text{MSE}}
\newcommand{\RMSE}{\text{RMSE}}
\newcommand{\SSR}{\text{SSR}}
\newcommand{\SST}{\text{SST}}
\newcommand{\JSest}{\delta_{\text{JS}}(\x)}
\newcommand{\Bayesest}{\delta_{\text{Bayes}}(\x)}
\newcommand{\EmpBayesest}{\delta_{\text{EmpBayes}}(\x)}
\newcommand{\BLUPest}{\delta_{\text{BLUP}}}
\newcommand{\MLEest}[1]{\hat{#1}_{\text{MLE}}}

%shortcuts for Linear Algebra stuff (i.e. vectors and matrices)
\newcommand{\twovec}[2]{\bracks{\begin{array}{c} #1 \\ #2 \end{array}}}
\newcommand{\threevec}[3]{\bracks{\begin{array}{c} #1 \\ #2 \\ #3 \end{array}}}
\newcommand{\fivevec}[5]{\bracks{\begin{array}{c} #1 \\ #2 \\ #3 \\ #4 \\ #5 \end{array}}}
\newcommand{\twobytwomat}[4]{\bracks{\begin{array}{cc} #1 & #2 \\ #3 & #4 \end{array}}}
\newcommand{\threebytwomat}[6]{\bracks{\begin{array}{cc} #1 & #2 \\ #3 & #4 \\ #5 & #6 \end{array}}}

%shortcuts for conventional compound symbols
\newcommand{\thetainthetas}{\theta \in \Theta}
\newcommand{\reals}{\mathbb{R}}
\newcommand{\complexes}{\mathbb{C}}
\newcommand{\rationals}{\mathbb{Q}}
\newcommand{\integers}{\mathbb{Z}}
\newcommand{\naturals}{\mathbb{N}}
\newcommand{\forallninN}{~~\forall n \in \naturals}
\newcommand{\forallxinN}[1]{~~\forall #1 \in \reals}
\newcommand{\matrixdims}[2]{\in \reals^{\,#1 \times #2}}
\newcommand{\inRn}[1]{\in \reals^{\,#1}}
\newcommand{\mathimplies}{\quad\Rightarrow\quad}
\newcommand{\mathlogicequiv}{\quad\Leftrightarrow\quad}
\newcommand{\eqncomment}[1]{\quad \text{(#1)}}
\newcommand{\limitn}{\lim_{n \rightarrow \infty}}
\newcommand{\limitN}{\lim_{N \rightarrow \infty}}
\newcommand{\limitd}{\lim_{d \rightarrow \infty}}
\newcommand{\limitt}{\lim_{t \rightarrow \infty}}
\newcommand{\limitsupn}{\limsup_{n \rightarrow \infty}~}
\newcommand{\limitinfn}{\liminf_{n \rightarrow \infty}~}
\newcommand{\limitk}{\lim_{k \rightarrow \infty}}
\newcommand{\limsupn}{\limsup_{n \rightarrow \infty}}
\newcommand{\limsupk}{\limsup_{k \rightarrow \infty}}
\newcommand{\floor}[1]{\left\lfloor #1 \right\rfloor}
\newcommand{\ceil}[1]{\left\lceil #1 \right\rceil}

%shortcuts for environments
\newcommand{\beqn}{\vspace{-0.25cm}\begin{eqnarray*}}
\newcommand{\eeqn}{\end{eqnarray*}}
\newcommand{\bneqn}{\vspace{-0.25cm}\begin{eqnarray}}
\newcommand{\eneqn}{\end{eqnarray}}

%shortcuts for mini environments
\newcommand{\parens}[1]{\left(#1\right)}
\newcommand{\squared}[1]{\parens{#1}^2}
\newcommand{\tothepow}[2]{\parens{#1}^{#2}}
\newcommand{\prob}[1]{\mathbb{P}\parens{#1}}
\newcommand{\cprob}[2]{\prob{#1~|~#2}}
\newcommand{\littleo}[1]{o\parens{#1}}
\newcommand{\bigo}[1]{O\parens{#1}}
\newcommand{\Lp}[1]{\mathbb{L}^{#1}}
\renewcommand{\arcsin}[1]{\text{arcsin}\parens{#1}}
\newcommand{\prodonen}[2]{\bracks{\prod_{#1=1}^n #2}}
\newcommand{\mysum}[4]{\sum_{#1=#2}^{#3} #4}
\newcommand{\sumonen}[2]{\sum_{#1=1}^n #2}
\newcommand{\infsum}[2]{\sum_{#1=1}^\infty #2}
\newcommand{\infprod}[2]{\prod_{#1=1}^\infty #2}
\newcommand{\infunion}[2]{\bigcup_{#1=1}^\infty #2}
\newcommand{\infinter}[2]{\bigcap_{#1=1}^\infty #2}
\newcommand{\infintegral}[2]{\int^\infty_{-\infty} #2 ~\text{d}#1}
\newcommand{\supthetas}[1]{\sup_{\thetainthetas}\braces{#1}}
\newcommand{\bracks}[1]{\left[#1\right]}
\newcommand{\braces}[1]{\left\{#1\right\}}
\newcommand{\set}[1]{\left\{#1\right\}}
\newcommand{\abss}[1]{\left|#1\right|}
\newcommand{\norm}[1]{\left|\left|#1\right|\right|}
\newcommand{\normsq}[1]{\norm{#1}^2}
\newcommand{\inverse}[1]{\parens{#1}^{-1}}
\newcommand{\rowof}[2]{\parens{#1}_{#2\cdot}}

%shortcuts for functionals
\newcommand{\realcomp}[1]{\text{Re}\bracks{#1}}
\newcommand{\imagcomp}[1]{\text{Im}\bracks{#1}}
\newcommand{\range}[1]{\text{range}\bracks{#1}}
\newcommand{\colsp}[1]{\text{colsp}\bracks{#1}}
\newcommand{\rowsp}[1]{\text{rowsp}\bracks{#1}}
\newcommand{\tr}[1]{\text{tr}\bracks{#1}}
\newcommand{\rank}[1]{\text{rank}\bracks{#1}}
\newcommand{\proj}[2]{\text{Proj}_{#1}\bracks{#2}}
\newcommand{\projcolspX}[1]{\text{Proj}_{\colsp{\X}}\bracks{#1}}
\newcommand{\median}[1]{\text{median}\bracks{#1}}
\newcommand{\mean}[1]{\text{mean}\bracks{#1}}
\newcommand{\dime}[1]{\text{dim}\bracks{#1}}
\renewcommand{\det}[1]{\text{det}\bracks{#1}}
\newcommand{\expe}[1]{\mathbb{E}\bracks{#1}}
\newcommand{\expeabs}[1]{\expe{\abss{#1}}}
\newcommand{\expesub}[2]{\mathbb{E}_{#1}\bracks{#2}}
\newcommand{\indic}[1]{\mathds{1}_{#1}}
\newcommand{\var}[1]{\mathbb{V}\text{ar}\bracks{#1}}
\newcommand{\cov}[2]{\mathbb{C}\text{ov}\bracks{#1, #2}}
\newcommand{\corr}[2]{\text{Corr}\bracks{#1, #2}}
\newcommand{\se}[1]{\mathbb{S}\text{E}\bracks{#1}}
\newcommand{\seest}[1]{\hat{\text{SE}}\bracks{#1}}
\newcommand{\bias}[1]{\text{Bias}\bracks{#1}}
\newcommand{\derivop}[2]{\dfrac{\text{d}}{\text{d} #1}\bracks{#2}}
\newcommand{\partialop}[2]{\dfrac{\partial}{\partial #1}\bracks{#2}}
\newcommand{\secpartialop}[2]{\dfrac{\partial^2}{\partial #1^2}\bracks{#2}}
\newcommand{\mixpartialop}[3]{\dfrac{\partial^2}{\partial #1 \partial #2}\bracks{#3}}

%shortcuts for functions
\renewcommand{\exp}[1]{\mathrm{exp}\parens{#1}}
\renewcommand{\cos}[1]{\text{cos}\parens{#1}}
\renewcommand{\sin}[1]{\text{sin}\parens{#1}}
\newcommand{\sign}[1]{\text{sign}\parens{#1}}
\newcommand{\are}[1]{\mathrm{ARE}\parens{#1}}
\newcommand{\natlog}[1]{\ln\parens{#1}}
\newcommand{\oneover}[1]{\frac{1}{#1}}
\newcommand{\overtwo}[1]{\frac{#1}{2}}
\newcommand{\overn}[1]{\frac{#1}{n}}
\newcommand{\oneoversqrt}[1]{\oneover{\sqrt{#1}}}
\newcommand{\sqd}[1]{\parens{#1}^2}
\newcommand{\loss}[1]{\ell\parens{\theta, #1}}
\newcommand{\losstwo}[2]{\ell\parens{#1, #2}}
\newcommand{\cf}{\phi(t)}

%English language specific shortcuts
\newcommand{\ie}{\textit{i.e.} }
\newcommand{\AKA}{\textit{AKA} }
\renewcommand{\iff}{\textit{iff}}
\newcommand{\eg}{\textit{e.g.} }
\newcommand{\st}{\textit{s.t.} }
\newcommand{\wrt}{\textit{w.r.t.} }
\newcommand{\mathst}{~~\text{\st}~~}
\newcommand{\mathand}{~~\text{and}~~}
\newcommand{\ala}{\textit{a la} }
\newcommand{\ppp}{posterior predictive p-value}
\newcommand{\dd}{dataset-to-dataset}

%shortcuts for distribution titles
\newcommand{\logistic}[2]{\mathrm{Logistic}\parens{#1,\,#2}}
\newcommand{\bernoulli}[1]{\mathrm{Bernoulli}\parens{#1}}
\newcommand{\betanot}[2]{\mathrm{Beta}\parens{#1,\,#2}}
\newcommand{\stdbetanot}{\betanot{\alpha}{\beta}}
\newcommand{\multnormnot}[3]{\mathcal{N}_{#1}\parens{#2,\,#3}}
\newcommand{\normnot}[2]{\mathcal{N}\parens{#1,\,#2}}
\newcommand{\classicnormnot}{\normnot{\mu}{\sigsq}}
\newcommand{\stdnormnot}{\normnot{0}{1}}
\newcommand{\uniformdiscrete}[1]{\mathrm{Uniform}\parens{\braces{#1}}}
\newcommand{\uniform}[2]{\mathrm{U}\parens{#1,\,#2}}
\newcommand{\stduniform}{\uniform{0}{1}}
\newcommand{\geometric}[1]{\mathrm{Geometric}\parens{#1}}
\newcommand{\hypergeometric}[3]{\mathrm{Hypergeometric}\parens{#1,\,#2,\,#3}}
\newcommand{\exponential}[1]{\mathrm{Exp}\parens{#1}}
\newcommand{\gammadist}[2]{\mathrm{Gamma}\parens{#1, #2}}
\newcommand{\poisson}[1]{\mathrm{Poisson}\parens{#1}}
\newcommand{\binomial}[2]{\mathrm{Binomial}\parens{#1,\,#2}}
\newcommand{\negbin}[2]{\mathrm{NegBin}\parens{#1,\,#2}}
\newcommand{\rayleigh}[1]{\mathrm{Rayleigh}\parens{#1}}
\newcommand{\multinomial}[2]{\mathrm{Multinomial}\parens{#1,\,#2}}
\newcommand{\gammanot}[2]{\mathrm{Gamma}\parens{#1,\,#2}}
\newcommand{\cauchynot}[2]{\text{Cauchy}\parens{#1,\,#2}}
\newcommand{\invchisqnot}[1]{\text{Inv}\chisq{#1}}
\newcommand{\invscaledchisqnot}[2]{\text{ScaledInv}\ncchisq{#1}{#2}}
\newcommand{\invgammanot}[2]{\text{InvGamma}\parens{#1,\,#2}}
\newcommand{\chisq}[1]{\chi^2_{#1}}
\newcommand{\ncchisq}[2]{\chi^2_{#1}\parens{#2}}
\newcommand{\ncF}[3]{F_{#1,#2}\parens{#3}}

%shortcuts for PDF's of common distributions
\newcommand{\logisticpdf}[3]{\oneover{#3}\dfrac{\exp{-\dfrac{#1 - #2}{#3}}}{\parens{1+\exp{-\dfrac{#1 - #2}{#3}}}^2}}
\newcommand{\betapdf}[3]{\dfrac{\Gamma(#2 + #3)}{\Gamma(#2)\Gamma(#3)}#1^{#2-1} (1-#1)^{#3-1}}
\newcommand{\normpdf}[3]{\frac{1}{\sqrt{2\pi#3}}\exp{-\frac{1}{2#3}(#1 - #2)^2}}
\newcommand{\normpdfvarone}[2]{\dfrac{1}{\sqrt{2\pi}}e^{-\half(#1 - #2)^2}}
\newcommand{\chisqpdf}[2]{\dfrac{1}{2^{#2/2}\Gamma(#2/2)}\; {#1}^{#2/2-1} e^{-#1/2}}
\newcommand{\invchisqpdf}[2]{\dfrac{2^{-\overtwo{#1}}}{\Gamma(#2/2)}\,{#1}^{-\overtwo{#2}-1}  e^{-\oneover{2 #1}}}
\newcommand{\exponentialpdf}[2]{#2\exp{-#2#1}}
\newcommand{\poissonpdf}[2]{\dfrac{e^{-#1} #1^{#2}}{#2!}}
\newcommand{\binomialpdf}[3]{\binom{#2}{#1}#3^{#1}(1-#3)^{#2-#1}}
\newcommand{\rayleighpdf}[2]{\dfrac{#1}{#2^2}\exp{-\dfrac{#1^2}{2 #2^2}}}
\newcommand{\gammapdf}[3]{\dfrac{#3^#2}{\Gamma\parens{#2}}#1^{#2-1}\exp{-#3 #1}}
\newcommand{\cauchypdf}[3]{\oneover{\pi} \dfrac{#3}{\parens{#1-#2}^2 + #3^2}}
\newcommand{\Gammaf}[1]{\Gamma\parens{#1}}

%shortcuts for miscellaneous typesetting conveniences
\newcommand{\notesref}[1]{\marginpar{\color{gray}\tt #1\color{black}}}

%%%% DOMAIN-SPECIFIC SHORTCUTS

%Real analysis related shortcuts
\newcommand{\zeroonecl}{\bracks{0,1}}
\newcommand{\forallepsgrzero}{\forall \epsilon > 0~~}
\newcommand{\lessthaneps}{< \epsilon}
\newcommand{\fraccomp}[1]{\text{frac}\bracks{#1}}

%Bayesian related shortcuts
\newcommand{\yrep}{y^{\text{rep}}}
\newcommand{\yrepisq}{(\yrep_i)^2}
\newcommand{\yrepvec}{\bv{y}^{\text{rep}}}


%Probability shortcuts
\newcommand{\SigField}{\mathcal{F}}
\newcommand{\ProbMap}{\mathcal{P}}
\newcommand{\probtrinity}{\parens{\Omega, \SigField, \ProbMap}}
\newcommand{\convp}{~{\buildrel p \over \rightarrow}~}
\newcommand{\convLp}[1]{~{\buildrel \Lp{#1} \over \rightarrow}~}
\newcommand{\nconvp}{~{\buildrel p \over \nrightarrow}~}
\newcommand{\convae}{~{\buildrel a.e. \over \longrightarrow}~}
\newcommand{\convau}{~{\buildrel a.u. \over \longrightarrow}~}
\newcommand{\nconvau}{~{\buildrel a.u. \over \nrightarrow}~}
\newcommand{\nconvae}{~{\buildrel a.e. \over \nrightarrow}~}
\newcommand{\convd}{~{\buildrel \mathcal{D} \over \rightarrow}~}
\newcommand{\nconvd}{~{\buildrel \mathcal{D} \over \nrightarrow}~}
\newcommand{\withprob}{~~\text{w.p.}~~}
\newcommand{\io}{~~\text{i.o.}}

\newcommand{\Acl}{\bar{A}}
\newcommand{\ENcl}{\bar{E}_N}
\newcommand{\diam}[1]{\text{diam}\parens{#1}}

\newcommand{\taua}{\tau_a}

\newcommand{\myint}[4]{\int_{#2}^{#3} #4 \,\text{d}#1}
\newcommand{\laplacet}[1]{\mathscr{L}\bracks{#1}}
\newcommand{\laplaceinvt}[1]{\mathscr{L}^{-1}\bracks{#1}}
\renewcommand{\min}[1]{\text{min}\braces{#1}}
\renewcommand{\max}[1]{\text{max}\braces{#1}}

\newcommand{\Vbar}[1]{\bar{V}\parens{#1}}
\newcommand{\expnegrtau}{\exp{-r\tau}}

%%% problem typesetting
\newcommand{\problem}{\noindent \colorbox{black}{{\color{yellow} \large{\textsf{\textbf{Problem \arabic{probnum}}}}~}} \addtocounter{probnum}{1} \vspace{0.2cm} \\ }

\newcommand{\easysubproblem}{\ingreen{\item} [easy] }
\newcommand{\intermediatesubproblem}{\inorange{\item} [harder] }
\newcommand{\hardsubproblem}{\inred{\item} [difficult] }
\newcommand{\extracreditsubproblem}{\inpurple{\item} [E.C.] }

\makeatletter
\newalphalph{\alphmult}[mult]{\@alph}{26}
\renewcommand{\labelenumi}{(\alphmult{\value{enumi}})}

\newcommand{\support}[1]{\text{Supp}\bracks{#1}}
\newcommand{\mode}[1]{\text{Mode}\bracks{#1}}
\newcommand{\IQR}[1]{\text{IQR}\bracks{#1}}
\newcommand{\quantile}[2]{\text{Quantile}\bracks{#1,\,#2}}



\newtoggle{spacingmode}
\begin{document}
\maketitle

\problem Denseness
\begin{enumerate}
\item Prove that $\mathbb{Q}$ is dense in $\mathbb{Q}$.  \\ \\
Let $x \in \mathbb{Q}$ and let $y \in \mathbb{Q}$. WLOG, assume that $x<y$. We want to show that there exists a $q \in \mathbb{Q}$ such that $x<q<y$. I claim that $q$ can be taken to be $\dfrac{x+y}{2}$. First note that by the closure property of the field $\mathbb{Q}$, we have $x+y \in \mathbb{Q}$ and $\dfrac{1}{2} \in \mathbb{Q}$, thus $\dfrac{x+y}{2} \in \mathbb{Q}$. We now must show that $x< \dfrac{x+y}{2}<y$. We will first show $x< \dfrac{x+y}{2}$. \begin{gather*}
      x<y \\
x+x < x+y  \\
2x < x + y \\
x < \dfrac{x+y}{2}
\end{gather*}

We must now show that  $\dfrac{x+y}{2}<y$.
\begin{gather*}
      x<y \\
x+y < y+y \\ 
x+y < 2y \\
\dfrac{x+y}{2} < y
\end{gather*}
Hence, $\mathbb{Q}$ is dense in $\mathbb{Q}$. \\ \\

\item Show that $\mathbb{R}$ is dense in $\mathbb{R}$. \\ \\

Let $x \in \mathbb{R}$ and let $y \in \mathbb{R}$ such that $x<y$. We want to show that there exists a $q \in \mathbb{R}$ such that $x<q<y$. By the same argument above, let us take $q=\dfrac{x+y}{2}$. By the closure property of $\mathbb{R}$ it is clear that $\dfrac{x+y}{2} \in \mathbb{R}$.  We will first show $x< \dfrac{x+y}{2}$. \begin{align*}
      x<y \\
x+x < x+y \\ 
2x < x + y \\
x < \dfrac{x+y}{2}
\end{align*}
We must now show that  $\dfrac{x+y}{2}<y$.
\begin{align*}
      x<y \\
x+y < y+y \\ 
x+y < 2y \\
\dfrac{x+y}{2} < y
\end{align*}
\\ \\


\item Prove that $\mathbb{R}$ is dense in $\mathbb{Q}$. \\ \\
Let $x \in \mathbb{Q}$ and let $y \in \mathbb{Q}$ such that $x<y$. We want to show that there exists a $q \in \mathbb{R}$ such that $x<q<y$. \\
We will make use of the Archimedean Property. Since $x<y$ this implies that $y-x>0$. Since $1$ is a real number, then we may invoke the Archimedean property. That is,there exists a positive integer $n$ such that $n(y-x)>1$. So we have
\begin{align*}
n(y-x)>1 \\
ny-nx>1
\end{align*} \begin{equation}
ny>nx+1
\end{equation}
Let us set Equation 1 aside and come back to it later. Let $m$ be the smallest integer such that $m>nx$. This implies that \begin{equation}
\dfrac{m}{n} > x
\end{equation}. \\

Since $m$ is the smallest integer such that $m>nx$, then $m-1 \leq nx$. To see this, note if $m-1 > nx$, then we would have $m> m-1 > nx$ which contradicts our choice of $m$. So $m-1 \leq nx$ implies $m \leq nx+1$. So Equation 1 and $m \leq nx+1$ implies
\begin{equation}
m \leq nx+1 < ny 
\end{equation}
\begin{equation}
m < ny 
\end{equation}
\begin{equation}
\dfrac{m}{n} < y
\end{equation}
Putting Eq.2 and Eq.5 together tells us
\begin{equation}
x < \dfrac{m}{n} < y 
\end{equation}
To complete the proof, we must show $\dfrac{m}{n}$ is a real number. Since $m$ and $n$ are defined to be integers and $n$ cannot be zero since $n$ is positive, then $\dfrac{m}{n} \in \mathbb{Q}$ which is a subset of $\mathbb{R}$ and hence $\dfrac{m}{n} \in \mathbb{R}$ .

\end{enumerate}

\problem 
\begin{enumerate}


\item Prove Rudin's Theorem 1.20 \\ \\


\textbf{Theorem 1.20} For every real $x>0$ and every integer $n>0$ there is one and only one positive real $y$ such that $y^{n}=x$ \\ 
  
\textbf{Proof}: We first will show uniqueness. Assume that there exists two distinct $y$'s, namely $y_{1}$ and $y_{2}$ such that $y_{1} ^{n} = x$ and $y_{2} ^{n} = x$. Since  $y_{1}$ and $y_{2}$ are distinct, we have either $y_{1} < y_{2}$ or $y_{1} > y_{2}$. WLOG, assume $y_{1} < y_{2}$. Then we have the following. \begin{align*}
y_{1} < y_{2} \\
y_{1} ^{n} < y_{2} ^{n} \\
x < x
\end{align*}
Contradiction! Thus, it must be the case that $y_{1}=y_{2}$ and so $y$ is unique. \\ \\


Now we will prove the existence of such a $y$. To do so, let $E = \braces{t \; |\; t>0, t \in \mathbb{R} \; \text{and} \; t^{n}<x}$. We will first show that $E$ is not empty. To show this, it suffices to take $t=\dfrac{x}{1+x}$. Note that for sufficiently large $x$, $\dfrac{x}{1+x}$ is close to $1$ but never equal to $1$. Moreover, since $x>0$, then the smallest $\dfrac{x}{1+x}$ can be is some number close to $0$. Thus, $0 \leq t < 1$. \\ \\ By Lemma 1, (see below) this implies $t^{n} \leq t < x$ and so there exists some $t \in E$. \\ \\ Now that we know $E$ is not empty, we will also show that $E$ is bounded above. I claim that $1+x$ is an upper bound of $E$. That is, for all $t \in E$, $t \leq x+1$. To see this, we will do proof by contradiction. Assume that $t > 1+x$, where $t \in E$. Then, by Lemma 2, we have $t^{n} \geq t > x$, but this implies that $t \notin E$ and we have reached our contradiction. This tells us that $E$ is a nonempty subset of $\mathbb{R}$ which is bounded above. By the LUB property of $\mathbb{R}$, the LUB of $E$ exists. Let $y=$ sup $E$.\\ \\


We want to show that $y^{n}=x$. Well, since we are in an ordered set, if we can show that both $y^{n}<x$ and $y^{n}>x$ fails to hold, then it must mean that $y^{n}=x$. In order to see this, we will use the result from Lemma 4, namely
\begin{equation}
b^{n}-a^{n} < (b-a)nb^{n-1} \; \text{where} \; 0<a<b.
\end{equation} \\ 


Case One: We will first show that $y^{n}<x$ leads to a contradiction. So assume $y^{n}<x$. Choose an $h$ such that $0<h<1$ and 
\begin{equation}
h< \dfrac{x-y^{n}}{n(y+1)^{n-1}}
\end{equation}
Note that the right hand side of the above equation is positive so it follows by the denseness of $\mathbb{R} \; \text{in} \; \mathbb{R}$ that such an $h$ exists. With regards to Equation 7, substitute in $a=y$ and $b=y+h$. It follows  that 
\begin{equation}
(y+h)^{n} - y^{n} < hn(y+h)^{n-1} < hn(y+1)^{n-1} < x-y^{n}
\end{equation}
Note the first inequality, $(y+h)^{n} - y^{n} < hn(y+h)^{n-1}$ is a result from Eq. 7 while $hn(y+h)^{n-1} < hn(y+1)^{n-1}$ follows since $h<1$. The last inequality, $hn(y+1)^{n-1} < x-y^{n}$ comes directly from Eq. 8.  \\ \\


So, by Eq. 9, we have $(y+h)^{n} - y^{n}< x-y^{n}$ so $(y+h)^{n} < x$, which shows $y+h \in E$. Since $y+h > y$, this contradicts the fact that $y$ is an upper bound of $E$. \\ \\


Case Two: We now show that $y^{n}>x$ fails to hold. Assume $y^{n}>x$. Let
\begin{equation}
k = \dfrac{y^{n}-x}{ny^{n-1}}
\end{equation}
Then we have $0<k<y$. Why is $0<k$? Well since $y^{n}>x$ the numerator of Equation 10 will always be bigger than 0. $0$ divided by any nonzero real number, as seen in the denominator, will still give us 0. Hence $0<k$. Moreover, we have $k<y$ since the ratio between the numerator and denominator of Equation 10 can be at most $y - \epsilon$ where $\epsilon$ is some small positive real number. To see this, let $x$ be any positive real number as defined before and the smallest $n$ can be is $n=1$. Then as $x$ approaches 0 from the right, $x=0$ but can never actually be $0$ based on the restriction of $x$. So this case tells us $k<y$. Now as $x$ and $n$ increases, the ratio between the numerator and denominator becomes smaller and so that ratio must be less than $y$. \\ \\

Now if $y-k \leq t$, then this implies that 
\begin{equation} 
y^{n}-t^{n} \leq y^{n} - (y-k)^{n} < kny^{n-1} = y^{n}-x
\end{equation}
Note that we get $y^{n}-t^{n} \leq y^{n} - (y-k)^{n}$ from using the assumption that $t \geq y-k$. From there, we use Equation 7, to show $y^{n} - (y-k)^{n} < kny^{n-1}$. Lastly, $kny^{n-1} = y^{n}-x$ stems from Equation 10. From the chain of inequalities, we see that $y^{n}-t^{n} < y^{n}-x$, which shows $t^{n} > x$ so $t \notin E$. This means our assumption was wrong, so for $t \in E$, we must have $y-k>t$. However this implies that $y-k$ is an upper bound of $E$. Since we assumed  $y$ is the least upper bound of $E$ and since $y-k<y$, we have reached our contradiction. \\ 


Thus, it must be the case that $y^{n}=x$. \\
\end{enumerate}
\textbf{Lemmas}
\begin{enumerate}
\item \textbf{Lemma 1}: Let $x$ be a real positive number, let $n$ be a positive integer and let $t$ have the following restrictions: $t=\dfrac{x}{1+x} \; \text{and} \; t^{n}<x$. Prove that $t^{n} \leq t < x$. \\ \\


\textbf{Proof}: We will first show $t^{n} \leq t$. Assume that $t^{n} > t$. Then, substituting in for $t$ yields 
\begin{align*}
\bigg (\dfrac{x}{1+x} \bigg )^{n} > \dfrac{x}{1+x} \\
\dfrac{x^{n}}{(1+x)^{n}}  >  \dfrac{x}{1+x} \\ \\
x^{n} > x(1+x)^{n-1} \\
\end{align*}
which is false since distributing and simplifying the right hand side will yield a $x^{n}$ term accompanied by positive real numbers.  \\ \\ 

We now show $t < x$ by contradiction. Assume that $t \geq x$. Then we have \begin{align*}
t \geq x \\
\dfrac{x}{1+x} \geq x \\ \\
x \geq x(1+x) \\
1 \geq 1+x \\
0 \geq x
\end{align*}
which is clearly false since $x$ is defined to be positive.\\ \\


\item \textbf{Lemma 2}: Let $x$ be a real positive number, let $n$ be a positive integer and let $t$ be such that $t>1+x$. Then $t^{n} \geq t > x$. \\ \\


\textbf{Proof}: We will first show $t^{n} \geq t$ by contradiction. Assume that $t^{n}<t$. Then this implies that $t^{n-1}<1$ which is false since $t>1+x$ and therefore $t>1$ so this contradicts $t^{n-1}<1$. \\ \\

We now will show that  $t > x$. Assume that $t \leq x$. Clearly $t$ cannot equal $x$ because this violates $t>1+x$. It is also clear that $t < x$ leads to a contradiction since this again violates $t>1+x$.\\ \\ 


\item \textbf{Lemma 3}: Let $a$ and $b$ be real numbers and let $n$ be a positive integer. Then,
\begin{equation}
b^{k}-a^{k} = (b-a)(b^{k-1} + b^{k-2}a + \ldots + a^{k-1})
\end{equation} \\ 

\textbf{Proof}: \\
\begin{equation*}
(b-a)(b^{k-1}+b^{k-2}a+ \ldots + a^{k-1})
\end{equation*}
\begin{align*}
= b^{k} + b^{k-1}a + b^{k-2}a^{2} + b^{k-3}a^{3} + \dots + ba^{k-1} - ab^{k-1} - a^{2}b^{k-2} - \ldots - a^{k-1}b-a^{k}
\end{align*}
\begin{equation*}
= b^{k} + (b^{k-1}a -ab^{k-1}) + \ldots + (ba^{k-1} - a^{k-1}b) - a^{k} 
\end{equation*}
\begin{equation*}
=b^{k} - a^{k}
\end{equation*}
\item \textbf{Lemma 4}: Let $0<a<b$. Then $b^{n}-a^{n} < (b-a)nb^{n-1}$ \\ \\


\textbf{Proof}: Since $b^{n}-a^{n}=(b-a)(b^{n-1} + b^{n-2}a + \ldots + a^{n-1})$, we must show that 
\begin{align*}
(b-a)(b^{n-1} + b^{n-2}a + \ldots + a^{n-1}) < (b-a)nb^{n-1}
\end{align*}
Since $(b-a)$ is on both sides, the problem reduces to showing that 
\begin{align*}
(b^{n-1} + b^{n-2}a + \ldots + a^{n-1})<nb^{n-1}
\end{align*}
Since  $0<a<b$, we have \begin{align*}
(b^{n-1} + b^{n-2}a + \ldots + a^{n-1})< (b^{n-1} + b^{n-2}b + \ldots + b^{n-1})
\end{align*}
But notice that the right hand side is precisely $nb^{n-1}$, since $(b^{n-1} + b^{n-2}b + \ldots + b^{n-1})= b^{n-1}+b^{n-1}+ \ldots + b^{n-1}$ where there are a total of n $b^{n-1}$'s and thus we have proved what we needed to show:
\begin{align*}
(b^{n-1} + b^{n-2}a + \ldots + a^{n-1})<nb^{n-1}
\end{align*}
\item \textbf{Lemma 5}: Suppose $r$ is a rational number, $x$ and $y$ are real numbers, and $r<x+y$. Then there are rational numbers $s<x$ and $t<y$ with $r<s+t<x+y$. \\ \\


\textbf{Proof}: Since $r<x+y$, then subtracting $y$ and adding $x$ to both sides yields $r<-y+x<2x)$. Dividing by 2 gives us $\frac{r-y+x}{2} < x$. By Theorem 1.20(b) in Rudin,there exists a rational number $s$ such that \begin{equation*} 
\frac{r-y+x}{2} <s < x. 
\end{equation*} Similarly, there exists a rational number $t$ such that 
\begin{equation*} 
\frac{r-x+y}{2} < t < y 
\end{equation*}
Adding the above equations yields $r<s+t<x+y$.
\end{enumerate}
\problem Prove the following corollary of Theorem 1.20
\begin{enumerate}
\item \textbf{Corollary}: If $a$ and $b$ are real numbers and $n$ is a positive integer, then 
\begin{align*}
(ab)^{1/n}	= a^{1/n}b^{1/n}
\end{align*} \\ \\
\textbf{Proof}: Let $\alpha = a^{1/n}$ and let $\beta = b^{1/n}$. Then,
\begin{align*}
\alpha^{n} = (a^{1/n})^{n} = (a^{1/n})(a^{1/n}) \ldots (a^{1/n}) = a^{1}
\end{align*}
Similarly, $\beta^{n} = b^{1}$. Note that \begin{align*}
ab = \alpha^{n} \beta^{n} = (\alpha \alpha \dots \alpha)(\beta \beta \ldots \beta) = (\alpha \beta)(\alpha \beta) \ldots (\alpha \beta) = (\alpha \beta)^{n}
\end{align*}
Since $ab=(\alpha \beta)^{n}$, then by Theorem 1.20, we have \begin{equation}
\alpha \beta = (ab)^{1/n}
\end{equation} 
Also, based on our definition of $\alpha$ and $\beta$, we have 
\begin{equation}
\alpha \beta = a^{1/n}b^{1/n}
\end{equation} 
By the uniqueness portion of Theorem 1.20, it therefore follows that 
\begin{align*}
(ab)^{1/n}	= a^{1/n}b^{1/n}
\end{align*} 
\end{enumerate}

\problem Rudin Chapter 1 Question 6
\begin{enumerate}
\item Fix $b>1$. If $m, n, p, q$ are integers, $n>0$, $q>0$, and $r=\dfrac{m}{n}=\dfrac{p}{q}$, prove that 
\begin{align*}
(b^{m})^{1/n}=(b^p)^{1/q}
\end{align*}
Hence it makes sense to define $b^{r}=(b^{m})^{1/n}$ \\ \\
First note that $\dfrac{m}{n}=\dfrac{p}{q}$ implies $mq=np$. We will use this fact in a few short moments. Let $y^{n}=b^{m}$. Using $y^{n}=b^{m}$, we have $y^{nq}=b^{mq}=b^{np}$, by our small fact mentioned above. As shown in the proof of the corollary of Theorem 1.20, we then may conclude that $(y^{q})^{n} = (b^{p})^{n}$. Since $n^{th}$ roots are unique, we see that $y^{q}=b^{p}$. Taking the $q^{th}$ roots of each side yields $y=(b^{p})^{1/q}$.  By a similar argument, we can show $y=(b^{m})^{1/n}$. To see this, we have shown $y^{nq}=b^{mq}$ so $(y^{n})^{q}=(b^{m})^{q}$. Taking the $q^{th}$ root yields $y^{n}=b^{m}$ and by taking the $n^{th}$ root, we have $y=(b^{m})^{1/n}$. Finally, since $y=(b^{p})^{1/q}$ and $y=(b^{m})^{1/n}$, by the uniqueness, we may conclude $(b^{m})^{1/n}=(b^p)^{1/q}$. Since this value is well-defined, that is, the representation does not matter, it is sensible to define   $b^{r}=(b^{m})^{1/n}$.  \\
\item  Prove that $b^{r+s} = b^{r}b^{s}$ if $r$ and $s$ are rational. \\ \\
To prove this, we first invoke the following lemma. \\
Lemma: If $b$ is a real number greater than 1, $x,y \in \mathbb{Z},$ then $b^{x}b^{y} = b^{x+y}.$ \\
Proof: $b^{x}b^{y} = (b_{x1} \times b_{x2} \ldots \times  b_{xx})(b_{y1} \times  b_{y2} \times  \ldots \times  b_{yy}) = b^{x+y}$ since $b$ is multiplied a total of $x+y$ times. \\ 

We now continue with the original proof. Since $r$ and $s$ are rational, then there exists integers $m,n,p,$ and $q$ such that $m,n,p,q$ are integers, $n \neq 0$, and $q \neq 0$, where $r=\dfrac{m}{n}$ and $s=\dfrac{p}{q}$. Note that $r+s=  \dfrac{m}{n} + \dfrac{p}{q} = \dfrac{mq+np}{nq}$. By part (a) and by the corollary to Theorem 1.20, we may write $b^{r+s}=(b^{mq+nq})^{1/nq} = (b^{mq}b^{np})^{1/nq} = (b^{mq})^{1/nq}(b^{np})^{1/nq} = (b^{m})^{1/n}(b^{p})^{1/q}=b^{r}b^{s}$. \\  \\ 

\item If $x$ is real, define $B(x)$ to be the set of all numbers $b^{t}$ where $t$ is rational and $t \leq x$. Prove that \begin{align*}
b^{r} = \; \text{sup} \; B(r)
\end{align*}
when $r$ is rational. Hence it makes sense to define 
\begin{align*}
b^{x}= \; \text{sup} \; B(r)
\end{align*}  \\ \\
We now must show that $ b^{r} =$ sup $B(r)$. In order to prove this, we must show two things: \\
\begin{enumerate}
\item We must show that $b^{r}$ is an upper bound  for $B(r)$, namely $b^{r} \geq x , \forall x \in B(r)$ 
\item We must show that any element less than $b^{r}$ fails to be an upper bound for $B(r)$. Namely if $b^{r} > \lambda$ then $\lambda$ fails to be a upper bound for $B(r)$.
\end{enumerate} 
Let us now prove the first claim. Let us denote the element $x \in B(r)$ by $b^{s}$. By definition, $s$ is rational and $ s \leq r$ so $0 \leq r-s$ which implies that $1 \leq b^{r-s}$. (Note $b>1$ by assumption). Multiplying by $b^{s}$ gives us $b^{s} \leq b^{r}$, so $b^{r}$ is an upper bound for $B(r)$. \\ 
To prove the second claim, note that $r \leq r$ and so $b^{r} \in B(r)$. So if $\lambda
< b^{r}$, then $\lambda$ fails to be an upper bound for $B(r)$. Thus, $ b^{r} =$ sup $B(r)$. \\ \\  
Since the numbers above are arbitrary, it holds for any such $r$ so it makes sense to define  $ b^{x} =$ sup $B(x)$ provided that $x$ is rational and $b>1$. \\ \\
\item Prove that $b^{x+y}=b^{x}b^{y}$ for all real $x$ and $y$. \\ \\
If we can somehow show that $b^{x+y} \leq b^{x}b^{y}$ and also show that $ b^{x}b^{y} \leq b^{x+y}$ then this would show $b^{x+y} = b^{x}b^{y}$. We will first attempt to show that $b^{x+y} \leq b^{x}b^{y}$. First note that $b^{x+y} =$ sup $B'(x+y)$ where $B'(z)=$ sup $\braces{b^{r} | r < z, r \in \mathbb{Q}}$. For every $b^{r} \in B'(x+y)$, we have $r<x+y$ so by Lemma 5, there exists rational numbers $s<x$ and $t<y$ such that $r<s+t<x+y$. By part (b) and by the fact that $b>1$, we have
\begin{align*}
b^{r}<b^{s+t}=b^{s+t}< b^{x+y}.
\end{align*}
This suggests that $b^{x}b^{y}$ is an upper bound for $B'(x+y)$, and so  $b^{x+y} \leq b^{x}b^{y}$. \\ \\
We now show $ b^{x}b^{y} \leq b^{x+y}$. If $r<x$ and $s<y$ then $r+s<x+y$. By part (b) and by using the fact that $b>1$ suggests $b^{x}b^{y}=$ sup $\braces{ b^{r}b^{s} | r<x, s<y, r, s \in \mathbb{Q}}$. This implies that $b^{x}b^{y} \leq b^{x+y}$. Thus $b^{x}b^{y} = b^{x+y}$.

\end{enumerate} 
\problem Rudin Question 7 Chapter 1. Fix $b>1$, $y>0$, and prove that there is a unique real $x$ such that $b^{x}=y$, bu completing the following outline. (This $x$ is called the logarithm of $y$ to the base $b$).
\begin{enumerate}


\item For any positive integer $n$, $b^{n}-1 \geq n(b-1)$. \\ \\
We will prove this by induction. For our base case, $n=1$, it is clear that $b-1 \geq b-1$ is a true statement. Assume this holds for $n=k$. We want to show that this holds for $k+1$. That is,
\begin{align*}
b^{k+1}-1 \geq (k+1)(b-1) \\
b^{k}b^{1}-1 \geq kb-b+b-1 \\
b^{k}b^{1}-b \geq kb-b \\
b(b^{k}-1) \geq k(b-1)
\end{align*}
which is true since $b^{k}-1 \geq k(b-1)$ by hypothesis and $b$ is a positive number greater than $1$. \\
\item Hence $b-1 \geq n(b^{1/n}-1)$. \\ \\
Note this follows immediately by setting $b$ in part (a) equal to $b^{1/n}$. This yields $b-1 \geq n(b^{1/n}-1)$ which is exactly what we needed to show. Note that the statement is still true because $b^{1/n}$ still satisfies the definition of $b$. \\ 


\item If $t>1$ and $n>(b-1)/(t-1)$, then $b^{1/n}<t$. \\ \\
Since $n>(b-1)/(t-1)$, then $(b-1)<n(t-1)$. By part (b) we may write $n(b^{1/n}-1) \leq (b-1)<n(t-1)$ and so $n(b^{1/n}-1) < n(t-1)$ which shows $(b^{1/n}-1) < t-1$ and thus we may conclude  $b^{1/n}<t$.

\item If $w$ is such that $b^{w} < y$, then $b^{w+(1/n)} < y$ for sufficiently large $n$. \\ \\
Let $t=\dfrac{y}{b^{w}}$. Then $t>1$. We already showed $b^{1/n}<t$ so multiplying on both sides by $b^{w}$ yields $b^{w}b^{1/n} < tb^{w}$ which shows $b^{w+(1/n)} < y$. \\ 

\item If $b^{w} > y$ then $b^{w-(1/n)}>y$ for sufficiently large $n$. \\ \\
Let $t=\dfrac{b^{w}}{y}$. Then $t>1$. By part (c) we have $b^{1/n}<t$. Dividing throughout by $b^{1/n}$ yields $1< \dfrac{t}{b^{1/n}}$. Substituting in for $t$ gives us $1< \dfrac{b^{w}}{yb^{1/n}}$ which gives us $y< b^{w-(1/n)}$. \\

\item Let $A$ be the set of all $w$ such that $b^{w}<y$ and show that $x=$ sup $A$ satisfies $b^{x}=y$. \\ \\
We will first show that sup $A$ exists. Note that since $b>1$, and $y>0$ we may choose a sufficiently small $w$ which will get us close to $0$. This tells us that $A$ is nonempty. Moreover, since $b^{w}<y$ for a sufficiently large $w$, $A$ is bounded above. Since we have a nonempty subset of $\mathbb{R}$ which is bounded above, then the supremum exists. Let sup $A = x$. \\ 
We must now show that $b^{x}=y$. To do this we will show that $b^{x}<y$ and $b^{x}>y$ lead to contradictions. \\ \\
Assume $b^{x}<y$. Then by part (d), $b^{x+(1/n)} < y$. This tells us that $x+(1/n) \in A$. However, $x$ is an upper bound of $A$ and since, $x+(1/n) > x$ we have reached our contradiction. \\
Assume $b^{x}>y$. Then by part (e) $b^{x-(1/n)}>y$. However, $x-(1/n) < x$ and $x$ is assumed to be the sup $A$ so every element less than $x$ fails to be an upper bound and thus we have reached our contradiction. \\
Therefore, it must be the case that $b^{x}=y$.

\item Prove that this $x$ is unique. \\ \\
Assume that $x$ is not unique. That is, there exists $x$ and $y$ such that $x=$ sup $A$ and $y=$ sup $A$, but $x \neq y$. Then, either $x>y$ or $x<y$. WLOG, assume that $x>y$. Then, be the definition of the supremum, every element less than $x$ fails to be an upper bound for $A$. However, $y$ is less than $x$ and $y$ is also a supremum and hence an upper bound as well. Contradiction!
\end{enumerate}
\end{document}