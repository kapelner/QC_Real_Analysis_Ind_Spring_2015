\documentclass[12pt]{article} 
\usepackage{amsmath}
\usepackage{amsfonts}
\usepackage{amssymb}
\usepackage{color}
\usepackage{enumerate}
\usepackage{gensymb}
\usepackage[hyperfootnotes=false]{hyperref}
\usepackage{graphicx}
\graphicspath{ {images/} }


\newtheorem{theorem}{Theorem}[section]
\newtheorem{corollary}{Corollary}[theorem]
\newtheorem{lemma}[theorem]{Lemma}
\title{Math 650.2  Problem Set 13}
\author{Elliot Gangaram\\
\date{}
\ elliot.gangaram@gmail.com \\}
%packages
%\usepackage{latexsym}
\usepackage{graphicx}
\usepackage{color}
\usepackage{amsmath}
\usepackage{dsfont}
\usepackage{placeins}
\usepackage{amssymb}
\usepackage{wasysym}
\usepackage{abstract}
\usepackage{hyperref}
\usepackage{etoolbox}
\usepackage{datetime}
\usepackage{xcolor}
\usepackage{alphalph}
\settimeformat{ampmtime}

%\usepackage{pstricks,pst-node,pst-tree}

%\usepackage{algpseudocode}
%\usepackage{amsthm}
%\usepackage{hyperref}
%\usepackage{mathrsfs}
%\usepackage{amsfonts}
%\usepackage{bbding}
%\usepackage{listings}
%\usepackage{appendix}
\usepackage[margin=1in]{geometry}
%\geometry{papersize={8.5in,11in},total={6.5in,9in}}
%\usepackage{cancel}
%\usepackage{algorithmic, algorithm}

\makeatletter
\def\maxwidth{ %
  \ifdim\Gin@nat@width>\linewidth
    \linewidth
  \else
    \Gin@nat@width
  \fi
}
\makeatother

\definecolor{fgcolor}{rgb}{0.345, 0.345, 0.345}
\newcommand{\hlnum}[1]{\textcolor[rgb]{0.686,0.059,0.569}{#1}}%
\newcommand{\hlstr}[1]{\textcolor[rgb]{0.192,0.494,0.8}{#1}}%
\newcommand{\hlcom}[1]{\textcolor[rgb]{0.678,0.584,0.686}{\textit{#1}}}%
\newcommand{\hlopt}[1]{\textcolor[rgb]{0,0,0}{#1}}%
\newcommand{\hlstd}[1]{\textcolor[rgb]{0.345,0.345,0.345}{#1}}%
\newcommand{\hlkwa}[1]{\textcolor[rgb]{0.161,0.373,0.58}{\textbf{#1}}}%
\newcommand{\hlkwb}[1]{\textcolor[rgb]{0.69,0.353,0.396}{#1}}%
\newcommand{\hlkwc}[1]{\textcolor[rgb]{0.333,0.667,0.333}{#1}}%
\newcommand{\hlkwd}[1]{\textcolor[rgb]{0.737,0.353,0.396}{\textbf{#1}}}%

\usepackage{framed}
\makeatletter
\newenvironment{kframe}{%
 \def\at@end@of@kframe{}%
 \ifinner\ifhmode%
  \def\at@end@of@kframe{\end{minipage}}%
  \begin{minipage}{\columnwidth}%
 \fi\fi%
 \def\FrameCommand##1{\hskip\@totalleftmargin \hskip-\fboxsep
 \colorbox{shadecolor}{##1}\hskip-\fboxsep
     % There is no \\@totalrightmargin, so:
     \hskip-\linewidth \hskip-\@totalleftmargin \hskip\columnwidth}%
 \MakeFramed {\advance\hsize-\width
   \@totalleftmargin\z@ \linewidth\hsize
   \@setminipage}}%
 {\par\unskip\endMakeFramed%
 \at@end@of@kframe}
\makeatother

\definecolor{shadecolor}{rgb}{.77, .77, .77}
\definecolor{messagecolor}{rgb}{0, 0, 0}
\definecolor{warningcolor}{rgb}{1, 0, 1}
\definecolor{errorcolor}{rgb}{1, 0, 0}
\newenvironment{knitrout}{}{} % an empty environment to be redefined in TeX

\usepackage{alltt}
\usepackage[T1]{fontenc}

\newcommand{\qu}[1]{``#1''}
\newcounter{probnum}
\setcounter{probnum}{1}

%create definition to allow local margin changes
\def\changemargin#1#2{\list{}{\rightmargin#2\leftmargin#1}\item[]}
\let\endchangemargin=\endlist 

%allow equations to span multiple pages
\allowdisplaybreaks

%define colors and color typesetting conveniences
\definecolor{gray}{rgb}{0.5,0.5,0.5}
\definecolor{black}{rgb}{0,0,0}
\definecolor{white}{rgb}{1,1,1}
\definecolor{blue}{rgb}{0.5,0.5,1}
\newcommand{\inblue}[1]{\color{blue}#1 \color{black}}
\definecolor{green}{rgb}{0.133,0.545,0.133}
\newcommand{\ingreen}[1]{\color{green}#1 \color{black}}
\definecolor{yellow}{rgb}{1,1,0}
\newcommand{\inyellow}[1]{\color{yellow}#1 \color{black}}
\definecolor{orange}{rgb}{0.9,0.649,0}
\newcommand{\inorange}[1]{\color{orange}#1 \color{black}}
\definecolor{red}{rgb}{1,0.133,0.133}
\newcommand{\inred}[1]{\color{red}#1 \color{black}}
\definecolor{purple}{rgb}{0.58,0,0.827}
\newcommand{\inpurple}[1]{\color{purple}#1 \color{black}}
\definecolor{backgcode}{rgb}{0.97,0.97,0.8}
\definecolor{Brown}{cmyk}{0,0.81,1,0.60}
\definecolor{OliveGreen}{cmyk}{0.64,0,0.95,0.40}
\definecolor{CadetBlue}{cmyk}{0.62,0.57,0.23,0}

%define new math operators
\DeclareMathOperator*{\argmax}{arg\,max~}
\DeclareMathOperator*{\argmin}{arg\,min~}
\DeclareMathOperator*{\argsup}{arg\,sup~}
\DeclareMathOperator*{\arginf}{arg\,inf~}
\DeclareMathOperator*{\convolution}{\text{\Huge{$\ast$}}}
\newcommand{\infconv}[2]{\convolution^\infty_{#1 = 1} #2}
%true functions

%%%% GENERAL SHORTCUTS

%shortcuts for pure typesetting conveniences
\newcommand{\bv}[1]{\boldsymbol{#1}}

%shortcuts for compound constants
\newcommand{\BetaDistrConst}{\dfrac{\Gamma(\alpha + \beta)}{\Gamma(\alpha)\Gamma(\beta)}}
\newcommand{\NormDistrConst}{\dfrac{1}{\sqrt{2\pi\sigma^2}}}

%shortcuts for conventional symbols
\newcommand{\tsq}{\tau^2}
\newcommand{\tsqh}{\hat{\tau}^2}
\newcommand{\sigsq}{\sigma^2}
\newcommand{\sigsqsq}{\parens{\sigma^2}^2}
\newcommand{\sigsqovern}{\dfrac{\sigsq}{n}}
\newcommand{\tausq}{\tau^2}
\newcommand{\tausqalpha}{\tau^2_\alpha}
\newcommand{\tausqbeta}{\tau^2_\beta}
\newcommand{\tausqsigma}{\tau^2_\sigma}
\newcommand{\betasq}{\beta^2}
\newcommand{\sigsqvec}{\bv{\sigma}^2}
\newcommand{\sigsqhat}{\hat{\sigma}^2}
\newcommand{\sigsqhatmlebayes}{\sigsqhat_{\text{Bayes, MLE}}}
\newcommand{\sigsqhatmle}[1]{\sigsqhat_{#1, \text{MLE}}}
\newcommand{\bSigma}{\bv{\Sigma}}
\newcommand{\bSigmainv}{\bSigma^{-1}}
\newcommand{\thetavec}{\bv{\theta}}
\newcommand{\thetahat}{\hat{\theta}}
\newcommand{\thetahatmle}{\hat{\theta}_{\mathrm{MLE}}}
\newcommand{\thetavechatmle}{\hat{\thetavec}_{\mathrm{MLE}}}
\newcommand{\muhat}{\hat{\mu}}
\newcommand{\musq}{\mu^2}
\newcommand{\muvec}{\bv{\mu}}
\newcommand{\muhatmle}{\muhat_{\text{MLE}}}
\newcommand{\lambdahat}{\hat{\lambda}}
\newcommand{\lambdahatmle}{\lambdahat_{\text{MLE}}}
\newcommand{\etavec}{\bv{\eta}}
\newcommand{\alphavec}{\bv{\alpha}}
\newcommand{\minimaxdec}{\delta^*_{\mathrm{mm}}}
\newcommand{\ybar}{\bar{y}}
\newcommand{\xbar}{\bar{x}}
\newcommand{\Xbar}{\bar{X}}
\newcommand{\phat}{\hat{p}}
\newcommand{\Phat}{\hat{P}}
\newcommand{\Zbar}{\bar{Z}}
\newcommand{\iid}{~{\buildrel iid \over \sim}~}
\newcommand{\inddist}{~{\buildrel ind \over \sim}~}
\newcommand{\approxdist}{~{\buildrel approx \over \sim}~}
\newcommand{\equalsindist}{~{\buildrel d \over =}~}
\newcommand{\loglik}[1]{\ell\parens{#1}}
\newcommand{\thetahatkminone}{\thetahat^{(k-1)}}
\newcommand{\thetahatkplusone}{\thetahat^{(k+1)}}
\newcommand{\thetahatk}{\thetahat^{(k)}}
\newcommand{\half}{\frac{1}{2}}
\newcommand{\third}{\frac{1}{3}}
\newcommand{\twothirds}{\frac{2}{3}}
\newcommand{\fourth}{\frac{1}{4}}
\newcommand{\fifth}{\frac{1}{5}}
\newcommand{\sixth}{\frac{1}{6}}

%shortcuts for vector and matrix notation
\newcommand{\A}{\bv{A}}
\newcommand{\At}{\A^T}
\newcommand{\Ainv}{\inverse{\A}}
\newcommand{\B}{\bv{B}}
\newcommand{\K}{\bv{K}}
\newcommand{\Kt}{\K^T}
\newcommand{\Kinv}{\inverse{K}}
\newcommand{\Kinvt}{(\Kinv)^T}
\newcommand{\M}{\bv{M}}
\newcommand{\Bt}{\B^T}
\newcommand{\Q}{\bv{Q}}
\newcommand{\Qt}{\Q^T}
\newcommand{\R}{\bv{R}}
\newcommand{\Rt}{\R^T}
\newcommand{\Z}{\bv{Z}}
\newcommand{\X}{\bv{X}}
\newcommand{\Xsub}{\X_{\text{(sub)}}}
\newcommand{\Xsubadj}{\X_{\text{(sub,adj)}}}
\newcommand{\I}{\bv{I}}
\newcommand{\Y}{\bv{Y}}
\newcommand{\sigsqI}{\sigsq\I}
\renewcommand{\P}{\bv{P}}
\newcommand{\Psub}{\P_{\text{(sub)}}}
\newcommand{\Pt}{\P^T}
\newcommand{\Pii}{P_{ii}}
\newcommand{\Pij}{P_{ij}}
\newcommand{\IminP}{(\I-\P)}
\newcommand{\Xt}{\bv{X}^T}
\newcommand{\XtX}{\Xt\X}
\newcommand{\XtXinv}{\parens{\Xt\X}^{-1}}
\newcommand{\XtXinvXt}{\XtXinv\Xt}
\newcommand{\XXtXinvXt}{\X\XtXinvXt}
\newcommand{\x}{\bv{x}}
\newcommand{\onevec}{\bv{1}}
\newcommand{\oneton}{1, \ldots, n}
\newcommand{\yoneton}{y_1, \ldots, y_n}
\newcommand{\yonetonorder}{y_{(1)}, \ldots, y_{(n)}}
\newcommand{\Yoneton}{Y_1, \ldots, Y_n}
\newcommand{\iinoneton}{i \in \braces{\oneton}}
\newcommand{\onetom}{1, \ldots, m}
\newcommand{\jinonetom}{j \in \braces{\onetom}}
\newcommand{\xoneton}{x_1, \ldots, x_n}
\newcommand{\Xoneton}{X_1, \ldots, X_n}
\newcommand{\xt}{\x^T}
\newcommand{\y}{\bv{y}}
\newcommand{\yt}{\y^T}
\renewcommand{\c}{\bv{c}}
\newcommand{\ct}{\c^T}
\newcommand{\tstar}{\bv{t}^*}
\renewcommand{\u}{\bv{u}}
\renewcommand{\v}{\bv{v}}
\renewcommand{\a}{\bv{a}}
\newcommand{\s}{\bv{s}}
\newcommand{\yadj}{\y_{\text{(adj)}}}
\newcommand{\xjadj}{\x_{j\text{(adj)}}}
\newcommand{\xjadjM}{\x_{j \perp M}}
\newcommand{\yhat}{\hat{\y}}
\newcommand{\yhatsub}{\yhat_{\text{(sub)}}}
\newcommand{\yhatstar}{\yhat^*}
\newcommand{\yhatstarnew}{\yhatstar_{\text{new}}}
\newcommand{\z}{\bv{z}}
\newcommand{\zt}{\z^T}
\newcommand{\bb}{\bv{b}}
\newcommand{\bbt}{\bb^T}
\newcommand{\bbeta}{\bv{\beta}}
\newcommand{\beps}{\bv{\epsilon}}
\newcommand{\bepst}{\beps^T}
\newcommand{\e}{\bv{e}}
\newcommand{\Mofy}{\M(\y)}
\newcommand{\KofAlpha}{K(\alpha)}
\newcommand{\ellset}{\mathcal{L}}
\newcommand{\oneminalph}{1-\alpha}
\newcommand{\SSE}{\text{SSE}}
\newcommand{\SSEsub}{\text{SSE}_{\text{(sub)}}}
\newcommand{\MSE}{\text{MSE}}
\newcommand{\RMSE}{\text{RMSE}}
\newcommand{\SSR}{\text{SSR}}
\newcommand{\SST}{\text{SST}}
\newcommand{\JSest}{\delta_{\text{JS}}(\x)}
\newcommand{\Bayesest}{\delta_{\text{Bayes}}(\x)}
\newcommand{\EmpBayesest}{\delta_{\text{EmpBayes}}(\x)}
\newcommand{\BLUPest}{\delta_{\text{BLUP}}}
\newcommand{\MLEest}[1]{\hat{#1}_{\text{MLE}}}

%shortcuts for Linear Algebra stuff (i.e. vectors and matrices)
\newcommand{\twovec}[2]{\bracks{\begin{array}{c} #1 \\ #2 \end{array}}}
\newcommand{\threevec}[3]{\bracks{\begin{array}{c} #1 \\ #2 \\ #3 \end{array}}}
\newcommand{\fivevec}[5]{\bracks{\begin{array}{c} #1 \\ #2 \\ #3 \\ #4 \\ #5 \end{array}}}
\newcommand{\twobytwomat}[4]{\bracks{\begin{array}{cc} #1 & #2 \\ #3 & #4 \end{array}}}
\newcommand{\threebytwomat}[6]{\bracks{\begin{array}{cc} #1 & #2 \\ #3 & #4 \\ #5 & #6 \end{array}}}

%shortcuts for conventional compound symbols
\newcommand{\thetainthetas}{\theta \in \Theta}
\newcommand{\reals}{\mathbb{R}}
\newcommand{\complexes}{\mathbb{C}}
\newcommand{\rationals}{\mathbb{Q}}
\newcommand{\integers}{\mathbb{Z}}
\newcommand{\naturals}{\mathbb{N}}
\newcommand{\forallninN}{~~\forall n \in \naturals}
\newcommand{\forallxinN}[1]{~~\forall #1 \in \reals}
\newcommand{\matrixdims}[2]{\in \reals^{\,#1 \times #2}}
\newcommand{\inRn}[1]{\in \reals^{\,#1}}
\newcommand{\mathimplies}{\quad\Rightarrow\quad}
\newcommand{\mathlogicequiv}{\quad\Leftrightarrow\quad}
\newcommand{\eqncomment}[1]{\quad \text{(#1)}}
\newcommand{\limitn}{\lim_{n \rightarrow \infty}}
\newcommand{\limitN}{\lim_{N \rightarrow \infty}}
\newcommand{\limitd}{\lim_{d \rightarrow \infty}}
\newcommand{\limitt}{\lim_{t \rightarrow \infty}}
\newcommand{\limitsupn}{\limsup_{n \rightarrow \infty}~}
\newcommand{\limitinfn}{\liminf_{n \rightarrow \infty}~}
\newcommand{\limitk}{\lim_{k \rightarrow \infty}}
\newcommand{\limsupn}{\limsup_{n \rightarrow \infty}}
\newcommand{\limsupk}{\limsup_{k \rightarrow \infty}}
\newcommand{\floor}[1]{\left\lfloor #1 \right\rfloor}
\newcommand{\ceil}[1]{\left\lceil #1 \right\rceil}

%shortcuts for environments
\newcommand{\beqn}{\vspace{-0.25cm}\begin{eqnarray*}}
\newcommand{\eeqn}{\end{eqnarray*}}
\newcommand{\bneqn}{\vspace{-0.25cm}\begin{eqnarray}}
\newcommand{\eneqn}{\end{eqnarray}}

%shortcuts for mini environments
\newcommand{\parens}[1]{\left(#1\right)}
\newcommand{\squared}[1]{\parens{#1}^2}
\newcommand{\tothepow}[2]{\parens{#1}^{#2}}
\newcommand{\prob}[1]{\mathbb{P}\parens{#1}}
\newcommand{\cprob}[2]{\prob{#1~|~#2}}
\newcommand{\littleo}[1]{o\parens{#1}}
\newcommand{\bigo}[1]{O\parens{#1}}
\newcommand{\Lp}[1]{\mathbb{L}^{#1}}
\renewcommand{\arcsin}[1]{\text{arcsin}\parens{#1}}
\newcommand{\prodonen}[2]{\bracks{\prod_{#1=1}^n #2}}
\newcommand{\mysum}[4]{\sum_{#1=#2}^{#3} #4}
\newcommand{\sumonen}[2]{\sum_{#1=1}^n #2}
\newcommand{\infsum}[2]{\sum_{#1=1}^\infty #2}
\newcommand{\infprod}[2]{\prod_{#1=1}^\infty #2}
\newcommand{\infunion}[2]{\bigcup_{#1=1}^\infty #2}
\newcommand{\infinter}[2]{\bigcap_{#1=1}^\infty #2}
\newcommand{\infintegral}[2]{\int^\infty_{-\infty} #2 ~\text{d}#1}
\newcommand{\supthetas}[1]{\sup_{\thetainthetas}\braces{#1}}
\newcommand{\bracks}[1]{\left[#1\right]}
\newcommand{\braces}[1]{\left\{#1\right\}}
\newcommand{\set}[1]{\left\{#1\right\}}
\newcommand{\abss}[1]{\left|#1\right|}
\newcommand{\norm}[1]{\left|\left|#1\right|\right|}
\newcommand{\normsq}[1]{\norm{#1}^2}
\newcommand{\inverse}[1]{\parens{#1}^{-1}}
\newcommand{\rowof}[2]{\parens{#1}_{#2\cdot}}

%shortcuts for functionals
\newcommand{\realcomp}[1]{\text{Re}\bracks{#1}}
\newcommand{\imagcomp}[1]{\text{Im}\bracks{#1}}
\newcommand{\range}[1]{\text{range}\bracks{#1}}
\newcommand{\colsp}[1]{\text{colsp}\bracks{#1}}
\newcommand{\rowsp}[1]{\text{rowsp}\bracks{#1}}
\newcommand{\tr}[1]{\text{tr}\bracks{#1}}
\newcommand{\rank}[1]{\text{rank}\bracks{#1}}
\newcommand{\proj}[2]{\text{Proj}_{#1}\bracks{#2}}
\newcommand{\projcolspX}[1]{\text{Proj}_{\colsp{\X}}\bracks{#1}}
\newcommand{\median}[1]{\text{median}\bracks{#1}}
\newcommand{\mean}[1]{\text{mean}\bracks{#1}}
\newcommand{\dime}[1]{\text{dim}\bracks{#1}}
\renewcommand{\det}[1]{\text{det}\bracks{#1}}
\newcommand{\expe}[1]{\mathbb{E}\bracks{#1}}
\newcommand{\expeabs}[1]{\expe{\abss{#1}}}
\newcommand{\expesub}[2]{\mathbb{E}_{#1}\bracks{#2}}
\newcommand{\indic}[1]{\mathds{1}_{#1}}
\newcommand{\var}[1]{\mathbb{V}\text{ar}\bracks{#1}}
\newcommand{\cov}[2]{\mathbb{C}\text{ov}\bracks{#1, #2}}
\newcommand{\corr}[2]{\text{Corr}\bracks{#1, #2}}
\newcommand{\se}[1]{\mathbb{S}\text{E}\bracks{#1}}
\newcommand{\seest}[1]{\hat{\text{SE}}\bracks{#1}}
\newcommand{\bias}[1]{\text{Bias}\bracks{#1}}
\newcommand{\derivop}[2]{\dfrac{\text{d}}{\text{d} #1}\bracks{#2}}
\newcommand{\partialop}[2]{\dfrac{\partial}{\partial #1}\bracks{#2}}
\newcommand{\secpartialop}[2]{\dfrac{\partial^2}{\partial #1^2}\bracks{#2}}
\newcommand{\mixpartialop}[3]{\dfrac{\partial^2}{\partial #1 \partial #2}\bracks{#3}}

%shortcuts for functions
\renewcommand{\exp}[1]{\mathrm{exp}\parens{#1}}
\renewcommand{\cos}[1]{\text{cos}\parens{#1}}
\renewcommand{\sin}[1]{\text{sin}\parens{#1}}
\newcommand{\sign}[1]{\text{sign}\parens{#1}}
\newcommand{\are}[1]{\mathrm{ARE}\parens{#1}}
\newcommand{\natlog}[1]{\ln\parens{#1}}
\newcommand{\oneover}[1]{\frac{1}{#1}}
\newcommand{\overtwo}[1]{\frac{#1}{2}}
\newcommand{\overn}[1]{\frac{#1}{n}}
\newcommand{\oneoversqrt}[1]{\oneover{\sqrt{#1}}}
\newcommand{\sqd}[1]{\parens{#1}^2}
\newcommand{\loss}[1]{\ell\parens{\theta, #1}}
\newcommand{\losstwo}[2]{\ell\parens{#1, #2}}
\newcommand{\cf}{\phi(t)}

%English language specific shortcuts
\newcommand{\ie}{\textit{i.e.} }
\newcommand{\AKA}{\textit{AKA} }
\renewcommand{\iff}{\textit{iff}}
\newcommand{\eg}{\textit{e.g.} }
\newcommand{\st}{\textit{s.t.} }
\newcommand{\wrt}{\textit{w.r.t.} }
\newcommand{\mathst}{~~\text{\st}~~}
\newcommand{\mathand}{~~\text{and}~~}
\newcommand{\ala}{\textit{a la} }
\newcommand{\ppp}{posterior predictive p-value}
\newcommand{\dd}{dataset-to-dataset}

%shortcuts for distribution titles
\newcommand{\logistic}[2]{\mathrm{Logistic}\parens{#1,\,#2}}
\newcommand{\bernoulli}[1]{\mathrm{Bernoulli}\parens{#1}}
\newcommand{\betanot}[2]{\mathrm{Beta}\parens{#1,\,#2}}
\newcommand{\stdbetanot}{\betanot{\alpha}{\beta}}
\newcommand{\multnormnot}[3]{\mathcal{N}_{#1}\parens{#2,\,#3}}
\newcommand{\normnot}[2]{\mathcal{N}\parens{#1,\,#2}}
\newcommand{\classicnormnot}{\normnot{\mu}{\sigsq}}
\newcommand{\stdnormnot}{\normnot{0}{1}}
\newcommand{\uniformdiscrete}[1]{\mathrm{Uniform}\parens{\braces{#1}}}
\newcommand{\uniform}[2]{\mathrm{U}\parens{#1,\,#2}}
\newcommand{\stduniform}{\uniform{0}{1}}
\newcommand{\geometric}[1]{\mathrm{Geometric}\parens{#1}}
\newcommand{\hypergeometric}[3]{\mathrm{Hypergeometric}\parens{#1,\,#2,\,#3}}
\newcommand{\exponential}[1]{\mathrm{Exp}\parens{#1}}
\newcommand{\gammadist}[2]{\mathrm{Gamma}\parens{#1, #2}}
\newcommand{\poisson}[1]{\mathrm{Poisson}\parens{#1}}
\newcommand{\binomial}[2]{\mathrm{Binomial}\parens{#1,\,#2}}
\newcommand{\negbin}[2]{\mathrm{NegBin}\parens{#1,\,#2}}
\newcommand{\rayleigh}[1]{\mathrm{Rayleigh}\parens{#1}}
\newcommand{\multinomial}[2]{\mathrm{Multinomial}\parens{#1,\,#2}}
\newcommand{\gammanot}[2]{\mathrm{Gamma}\parens{#1,\,#2}}
\newcommand{\cauchynot}[2]{\text{Cauchy}\parens{#1,\,#2}}
\newcommand{\invchisqnot}[1]{\text{Inv}\chisq{#1}}
\newcommand{\invscaledchisqnot}[2]{\text{ScaledInv}\ncchisq{#1}{#2}}
\newcommand{\invgammanot}[2]{\text{InvGamma}\parens{#1,\,#2}}
\newcommand{\chisq}[1]{\chi^2_{#1}}
\newcommand{\ncchisq}[2]{\chi^2_{#1}\parens{#2}}
\newcommand{\ncF}[3]{F_{#1,#2}\parens{#3}}

%shortcuts for PDF's of common distributions
\newcommand{\logisticpdf}[3]{\oneover{#3}\dfrac{\exp{-\dfrac{#1 - #2}{#3}}}{\parens{1+\exp{-\dfrac{#1 - #2}{#3}}}^2}}
\newcommand{\betapdf}[3]{\dfrac{\Gamma(#2 + #3)}{\Gamma(#2)\Gamma(#3)}#1^{#2-1} (1-#1)^{#3-1}}
\newcommand{\normpdf}[3]{\frac{1}{\sqrt{2\pi#3}}\exp{-\frac{1}{2#3}(#1 - #2)^2}}
\newcommand{\normpdfvarone}[2]{\dfrac{1}{\sqrt{2\pi}}e^{-\half(#1 - #2)^2}}
\newcommand{\chisqpdf}[2]{\dfrac{1}{2^{#2/2}\Gamma(#2/2)}\; {#1}^{#2/2-1} e^{-#1/2}}
\newcommand{\invchisqpdf}[2]{\dfrac{2^{-\overtwo{#1}}}{\Gamma(#2/2)}\,{#1}^{-\overtwo{#2}-1}  e^{-\oneover{2 #1}}}
\newcommand{\exponentialpdf}[2]{#2\exp{-#2#1}}
\newcommand{\poissonpdf}[2]{\dfrac{e^{-#1} #1^{#2}}{#2!}}
\newcommand{\binomialpdf}[3]{\binom{#2}{#1}#3^{#1}(1-#3)^{#2-#1}}
\newcommand{\rayleighpdf}[2]{\dfrac{#1}{#2^2}\exp{-\dfrac{#1^2}{2 #2^2}}}
\newcommand{\gammapdf}[3]{\dfrac{#3^#2}{\Gamma\parens{#2}}#1^{#2-1}\exp{-#3 #1}}
\newcommand{\cauchypdf}[3]{\oneover{\pi} \dfrac{#3}{\parens{#1-#2}^2 + #3^2}}
\newcommand{\Gammaf}[1]{\Gamma\parens{#1}}

%shortcuts for miscellaneous typesetting conveniences
\newcommand{\notesref}[1]{\marginpar{\color{gray}\tt #1\color{black}}}

%%%% DOMAIN-SPECIFIC SHORTCUTS

%Real analysis related shortcuts
\newcommand{\zeroonecl}{\bracks{0,1}}
\newcommand{\forallepsgrzero}{\forall \epsilon > 0~~}
\newcommand{\lessthaneps}{< \epsilon}
\newcommand{\fraccomp}[1]{\text{frac}\bracks{#1}}

%Bayesian related shortcuts
\newcommand{\yrep}{y^{\text{rep}}}
\newcommand{\yrepisq}{(\yrep_i)^2}
\newcommand{\yrepvec}{\bv{y}^{\text{rep}}}


%Probability shortcuts
\newcommand{\SigField}{\mathcal{F}}
\newcommand{\ProbMap}{\mathcal{P}}
\newcommand{\probtrinity}{\parens{\Omega, \SigField, \ProbMap}}
\newcommand{\convp}{~{\buildrel p \over \rightarrow}~}
\newcommand{\convLp}[1]{~{\buildrel \Lp{#1} \over \rightarrow}~}
\newcommand{\nconvp}{~{\buildrel p \over \nrightarrow}~}
\newcommand{\convae}{~{\buildrel a.e. \over \longrightarrow}~}
\newcommand{\convau}{~{\buildrel a.u. \over \longrightarrow}~}
\newcommand{\nconvau}{~{\buildrel a.u. \over \nrightarrow}~}
\newcommand{\nconvae}{~{\buildrel a.e. \over \nrightarrow}~}
\newcommand{\convd}{~{\buildrel \mathcal{D} \over \rightarrow}~}
\newcommand{\nconvd}{~{\buildrel \mathcal{D} \over \nrightarrow}~}
\newcommand{\withprob}{~~\text{w.p.}~~}
\newcommand{\io}{~~\text{i.o.}}

\newcommand{\Acl}{\bar{A}}
\newcommand{\ENcl}{\bar{E}_N}
\newcommand{\diam}[1]{\text{diam}\parens{#1}}

\newcommand{\taua}{\tau_a}

\newcommand{\myint}[4]{\int_{#2}^{#3} #4 \,\text{d}#1}
\newcommand{\laplacet}[1]{\mathscr{L}\bracks{#1}}
\newcommand{\laplaceinvt}[1]{\mathscr{L}^{-1}\bracks{#1}}
\renewcommand{\min}[1]{\text{min}\braces{#1}}
\renewcommand{\max}[1]{\text{max}\braces{#1}}

\newcommand{\Vbar}[1]{\bar{V}\parens{#1}}
\newcommand{\expnegrtau}{\exp{-r\tau}}

%%% problem typesetting
\newcommand{\problem}{\noindent \colorbox{black}{{\color{yellow} \large{\textsf{\textbf{Problem \arabic{probnum}}}}~}} \addtocounter{probnum}{1} \vspace{0.2cm} \\ }

\newcommand{\easysubproblem}{\ingreen{\item} [easy] }
\newcommand{\intermediatesubproblem}{\inorange{\item} [harder] }
\newcommand{\hardsubproblem}{\inred{\item} [difficult] }
\newcommand{\extracreditsubproblem}{\inpurple{\item} [E.C.] }

\makeatletter
\newalphalph{\alphmult}[mult]{\@alph}{26}
\renewcommand{\labelenumi}{(\alphmult{\value{enumi}})}

\newcommand{\support}[1]{\text{Supp}\bracks{#1}}
\newcommand{\mode}[1]{\text{Mode}\bracks{#1}}
\newcommand{\IQR}[1]{\text{IQR}\bracks{#1}}
\newcommand{\quantile}[2]{\text{Quantile}\bracks{#1,\,#2}}



\newtoggle{spacingmode}
\begin{document}
\maketitle

\problem Is every point of every open set $E \subset \mathbb{R}^{2}$ a limit point of $E$? Answer the same question for closed sets in $\mathbb{R}^{2}.$ \\

We first take care of the easy part, namely is every point of every closed set $E \subset \mathbb{R}^{2}$ a limit point of $E$? Here is a simple counter example. Consider the set $E= \braces{(0,0)}$. This set is closed because every limit point of $E$ belongs to $E$ since the set of limit points is just the empty set (Corollary to Theorem 2.20) and the empty set is a subset of every set. Therefore, $E$ is closed. However, $(0,0)$ is not a limit point for $E$ otherwise this would contradict the corollary to Theorem 2.20. \\ 

We now ask ourselves is every point of every open set $E \subset \mathbb{R}^{2}$ a limit point of $E$? Let $E$ be an open set in $\mathbb{R}^{2}$ and let $p \in E$. Then $p$ is an interior point meaning that there exists at least one neighborhood $N_{r}(p) \subset E$. Additionally, we know that $N_{r}(p)$ contains infinitely many points of $E$ (Theorem 2.20). We want to show that $p$ is a limit point of $E$, that is every neighborhood of $p$ contains a point $q \neq p$ such that $q \in E$. We have two cases, either the radius of the neighborhood of $p$ is smaller than or greater than the radius of an arbitrary neighborhood. \\

Case One: Let $r \leq s$. Then $N_{r}(p) \subseteq N_{s}(p)$. Since $N_{r}(p)$ contains some distinct point $q \in E$, (because $p$ is a limit point) then $q \in N_{s}(p)$ and so $p$ is a limit point of $E$ if $r \leq s$. \\

Case Two: We now assume that $r>s$. Well $N_{s}(p)$ contains infinitely many points by Theorem 2.20. However, we know something specific about these points. We have $N_{s}(p) \subset N_{r}(p) \subset E$, and so $N_{s}(p)$ contains infinitely many points of $E$ and thus $p$ is a limit point of $E$ since we have shown that for all neighborhoods of $p$ we can find at least one point $q \neq p$ such that $q \in E$. \\ \\

\problem Let $E^{\degree}$ denote the set of all interior points of a set $E$.
\begin{enumerate}
\item Prove that $E^{\degree}$ is always open. \\ 

Let $p \in E^{\degree}$. We wish to show that $p$ is an interior point of $E^{\degree}$. Since $p \in E^{\degree}$, then $p$ is an interior point of $E$. That means there exists some neighborhood, $N_{r}(p)$ such that $N_{r}(p) \subset E$. By Theorem 2.19, we know that $N_{r}(p)$ is open. That means if $q \in N_{r}(p)$, then there exists some $N_{s}(q)$ such that $N_{s}(q) \subset N_{r}(p) \subset E$. This shows that $q$ is an interior point of $E$ and so $q \in E^{\degree}$. Since $q$ was arbitrary, we see that all points in $N_{r}(p)$ are interior points of $E$ so $N_{r}(p) \subset E^{\degree}$. Thus we have shown that every point, $p \in E^{\degree}$ is an interior point of $E^{\degree}$ since we have shown for every point in $E^{\degree}$, there exists a neighborhood $N_{r}(p) \subset E^{\degree}$ which proves $E^{\degree}$ is open. \\

\item Prove that $E$ is open if and only if $E^{\degree}=E.$ \\ 

If $E^{\degree}=E$, then $E$ is open by part $(a)$. So assume that $E$ is open. We wish to show that $E^{\degree}=E.$ Since $E$ is open, if $p \in E$, then there exists some $N_{r}(p) \subset E$ and so $p \in E^{\degree}$. This shows that $E \subseteq E^{\degree}$. Conversely, let $p \in E^{\degree}$. Then $p$ is an interior point of $E$ meaning there exists some neighborhood $N_{r}(p)$ such that $N_{r}(p) \subset E$. Since $p \in N_{r}(p)$, then $p \in E$ which shows that $E^{\degree} \subseteq E$. Thus we have shown both sets are subsets of each other which implies that $E^{\degree}=E.$ \\

\item If $G \subset E$ and $G$ is open, prove that $G \subset E^{\degree}$.\\

Let $p \in G$. Since $G$ is open, $p$ is an interior point of $G$ meaning that we can find some neighborhood $N_{r}(p)$ such that $N_{r}(p) \subset G \subset E$. This tells us that every point in $G$ is also an interior point of $E$ since $N_{r}(p) \subset E$. So $p \in E^{\degree}$ which shows $G \subset E^{\degree}$. \\  

\item Prove that the complement of $E^{\degree}$ is the closure of the complement of $E$. \\ 

We must show that $(E^{\degree})^{c}=\bar{E^{c}}$. As usual, we will show set equality. \\ 

Let $p \in (E^{\degree})^{c}$. Then $p \notin E^{\degree}$ so that means $p$ is not an interior point of $E$. This means that for all neighborhoods, the intersection of $N_{r}(p)$ and $E^{c}$ cannot be empty since all neighborhoods of $p$ are not wholly contained in $E$. So we have two cases, either $p \in E^{c}$ or there exists some $x$ where $p \neq x$ such that $x \in N_{r}(p) \cap E^{c}$. \\

Case One: If $p \in E^{c}$ then $p \in  E^{c} \cup (E^{c})' = \bar{E^{c}}$ and thus $(E^{\degree})^{c} \subset \bar{E^{c}}$. \\

Case Two: Now assume that there exists some $x$ where $p \neq x$ such that $x \in N_{r}(p) \cap E^{c}$. Then $p \in (E^{c})'$ since we have shown $p$ is a limit point of $E^{c}$. So $p \in  E^{c} \cup (E^{c})' = \bar{E^{c}}$ and thus $(E^{\degree})^{c} \subset \bar{E^{c}}$. \\

By reversing the proof above we can show that $\bar{E^{c}} \subset (E^{\degree})^{c}$. Let $p \in \bar{E^{c}}$. Then $p \in E^{c}$ or $p \in (E^{c})'$. \\ 

Case One: If $p \in E^{c}$, then $p \notin E$ so $p$ is not an interior point of $E$ because all neighborhoods $N_{r}(p)$ are not wholly contained in $E$ since $p \notin E$. Thus $p \notin E^{\degree}$ and so $p \in (E^{\degree})^{c}$. \\

Case Two: If $p \in (E^{c})'$, then $p$ is a limit point of $E^{c}$ so every neighborhood of $p$ contains some point $q \in E^{c}$. Therefore, $p$ cannot be an interior point of $E$ so $p \notin E^{\degree}$ which implies that $p \in (E^{\degree})^{c}.$  \\ 

\item Do $E$ and $\bar{E}$ always have the same interiors? \\ 

$E$ and $\bar{E}$ do not always have the same interiors. Let $E= \mathbb{Q}$ in $\mathbb{R}^{1}$. Then $\mathbb{Q}$ has no interior points because any neighborhood of a rational number contains irrational numbers since we proved set of irrationals numbers are dense in $\mathbb{R}$ by a previous homework exercise. Thus $\mathbb{Q}^{\degree} = \emptyset$. \\ 

Now consider $\bar{E}=\bar{\mathbb{Q}}= \mathbb{Q} \cup \mathbb{Q}'$ where $\mathbb{Q}'$ denotes the set of all limit points of $\mathbb{Q}$. We see that $\mathbb{Q}'$ is the set of irrationals since any neighborhood around any irrational number contains a rational. Thus, $\bar{\mathbb{Q}}$ equals to the union of the rationals and irrationals which tells us that $\bar{E}= \mathbb{R}$. Finally, $(\bar{E})^{\degree} = (\mathbb{R})^{\degree} = \mathbb{R}$ because $\mathbb{R}$ is open in itself. So we see that $\mathbb{Q}^{\degree} = \emptyset  \neq (\bar{\mathbb{Q}})^{\degree} = \mathbb{R}$. \\ \\

\item Do $E$ and $E^{\degree}$ always have the same closures? \\

Recall that the closure of $A$ is the set $\bar{A}= A \cup A'$. Again letting $E = \mathbb{Q}$, we see that $(\bar{\mathbb{Q}})= \mathbb{R}$ by the argument above. Additionally, the closure of $E^{\degree}$ is the closure of $\mathbb{Q}^{\degree}$ which is $\bar{\mathbb{Q}^{\degree}}= \bar{\emptyset}= \emptyset$ and so they do not always have the same closure. \\
\end{enumerate}


\problem Let $X$ be an infinite set. For $p \in X$ and $q \in X$, defined 
\begin{equation*}
d(p,q) = \begin{cases}
             1  & \text{if } p \neq q \\
             0  & \text{if } p=q
       \end{cases} \quad
\end{equation*}
Prove that this is metric. Which subsets of the resulting metric space are open? Which are closed? \\

We first show that $d$ is metric defined on $X$.
\begin{enumerate}
\item It is clear that $d$ is a real valued function.
\item We must show that $d(p,q)>0$ if $p \neq q$. However, this follows directly from our definition of $d$ since $d(p,q)=1$ if $p \neq q$.
\item We must show that $d(p,p)=0$ but this follows directly from our definition of $d$.
\item We must show that $d(p,q)=d(q,p)$. We have two cases. If $p \neq q$, then we have $d(p,q)=1=d(q,p)$. Alternatively, if $p=q$, then we have $d(p,q)=0=d(q,p)$.
\item We must show that $d(p,q) \leq d(r,p)+d(r,q)$ for any $r \in X$. Note that we have two cases. Either $p=q$ or $p \neq q$. If $p=q$ then $d(p,q)=0$ but the right hand side is some nonnegative positive real number and thus $0 \leq d(r,p)+d(r,q)$ so the claim holds. Alternatively, assume that $p \neq q$. Note that in this case, if $p=r$ and $r=q$ then we would have violated the fact that $p \neq q$. So WLOG, assume that $p \neq r$. Then we have $d(p,q) = 1 \leq 1+d(r,q) = d(r,p)+d(r,q)$ and since $d(r,q)$ is nonnegative, the claim holds. This proves that $d$ is a metric. \\ 

\end{enumerate}

We now wish to consider which sets are open. Recall that an open set, $E$, is defined to be any set in the metric space such that every point of $E$ is an interior point of $E$. With the metric defined on $X$, it appears that any subset of $X$ is open. To see this let $E$ be an arbitrary subset of $X$. We have two cases. \\ 

In the case that $E= \emptyset$, then $E^{\degree} = \emptyset$ and so $E^{\degree}=E$ which shows that every limit point of $E$ is a point in $E$ and so $E$ is open. \\

Now let $E$ be any nonempty subset of $X$ and let $p \in E$. Observe what happens when we consider the neighborhood $N_{1/2}(p)$. $N_{1/2}(p)= \braces{x \in X ~|~d(x,p) < 1/2}.$ But by the definition of $d$, we have $N_{1/2}(p)= \braces{x \in X ~|~d(x,p) < 1/2} = \braces{p}.$ Observe that the limit points of $\braces{p}$ is the empty set by the corollary to Theorem 2.20 and so the set of limit points belongs to $N_{1/2}(p)$. Note that $N_{1/2}(p) = \braces{p} \subset E$. Therefore every point of $E$ is an interior point of $E$ since we have shown for every point in $E$, there exists some neighborhood $N_{1/2}(p)$ which is wholly contained in $E$. This is precisely the requirements to show that a set $E$ is open. \\ 

We now consider which sets are closed. We know that every subset of $X$ is open. By Theorem 2.23, we know that a set is open if and only if its complement is closed. Since all sets are open, it follows that all sets are closed. This is because if we take $A$ to be any subset of $X$, we may set $A=E^{c}= X - E$. Thus, by conditioning on $E$, we can get any subset of $X$. Hence all subsets of $X$ are closed. \\ \\


\problem Prove that the Cantor set is uncountable. \\ 

Let us first recap how the Cantor set is constructed. Let $E_{0}$ be the interval $[0,1]$. Remove the segment $(1/3, 2/3)$ and let $E_{1}$ be the union of intervals $[0, 1/3]$ and $[2/3, 1]$. Remove the middle thirds of these intervals and let $E_{2}$ be the union of the intervals $[0 , 1/9 ], [2/3 , 3/9], [6/9 , 7/9],$ and $[8/9 , 1]$. Continuing this way we obtain a sequence of sets $E_{n}$ such that $E_{1} \supset E_{2} \supset E_{3} \supset \ldots .$ The set $P = \cap_{n=1}^{\infty} E_{n}$ is called the Cantor set. On the next page, we graphically show the first few iterations of the Cantor Set. We see that a point in the Cantor set is uniquely determined as an infinite sequence of $L's$ and $R's.$ Alternatively, we can let $L=0$ and $R=1$, so any point in the Cantor set may be described as a unique infinite sequence of 0's and 1's. Let us assume that the Cantor set is countable. Then we can list the elements as follows:
\begin{center}
$a_{1} = a_{11}a_{12}a_{13}\ldots \ldots \ldots$  \\
$a_{2} = a_{21}a_{22}a_{23}\ldots \ldots \ldots$  \\
$a_{3} = a_{31}a_{32}a_{33}\ldots \ldots \ldots$  \\
\ldots \ldots \ldots \ldots \ldots \ldots  \ldots \\
\ldots \ldots \ldots \ldots \ldots \ldots  \ldots
\end{center} 
where $a_{ij}=0$ or 1. Moreover, since the Cantor set, $P$ is countable, there exists a bijection $f: P \rightarrow \mathbb{N}$ where $a_{n}$ maps to the natural number $n$. If we can show that there exists some number in the Cantor set which is missed by the general correspondence, then it shows that $P$ is not countable because $f$ is not a function on $P$. Let us attempt to construct this number which is missed by $f$. Call this number 
\begin{align*}
b_{k} = \begin{cases}
             0  & \text{if } a_{kk}=1  \\
             1  & \text{if } a_{kk}=0
       \end{cases} \quad
\end{align*}
Let us look along the diagonal of our list of elements. Notice that $b_{k}$ is no where in our list. To see this, assume that $b_{k}=a_{n}$ for some $n \in \mathbb{N}$. Then the $a_{nn}$ term in $a_{n}$ differs from the $b_{k}$ term and since a binary representation is unique, these two numbers are different. Thus $f$ is not a bijection, and so the Cantor set is uncountable. This reasoning is the same train of thought used to prove Theorem 2.14 which shows that the Cantor set is uncountable. Namely, let $A$ be the set of all sequences whose elements are the digits 0 and 1. This set $A$ is uncountable and the proof is exactly the proof given above. \\

\centerline{\includegraphics[scale=0.70]{cantor}}

\end{document}