\documentclass[12pt]{article} 
\usepackage{amsmath}
\usepackage{amsfonts}
\usepackage{amssymb}
\usepackage{color}

\newtheorem{theorem}{Theorem}[section]
\newtheorem{corollary}{Corollary}[theorem]
\newtheorem{lemma}[theorem]{Lemma}
\title{Math 650.2 Homework 3}
\author{Elliot Gangaram\\
\date{}
\ elliot.gangaram@gmail.com \\}
\include{preamble}


\newtoggle{spacingmode}
\begin{document}
\maketitle

\problem Prove propositions $1.18c-e$ regarding ordered fields.
\begin{enumerate}
\item Proposition $1.18c$ asserts the following: If $x<0$ and $y<z$ then $xy>xz$. \\ \\
Since $y<z$ it follows from Definition 1.17(i) that $y-y<z-y$ and hence $0<z-y$. We are also given that $x<0$ so it follows from Proposition 1.18$(a)$ that $(-x)>0$. By Definition 1.17(ii) we may write $(-x)(z-y)>0$ and again by 1.18$(a)$ this implies that $x(z-y)<0$. As a result of  1.17(i), this implies that \begin{gather*} 
x(z-y)+ xy < 0+ xy \\
xz-xy+xy<xy \\ 
xz<xy \\
xy>xz
\end{gather*} 
\item Proposition $1.18d$ asserts the following: If $x \neq 0$ then $x^{2}>0$. In particular, $1>0$. \\ \\
Since we are in an ordered set, we have two cases. Either $x>0$ or $x<0$. \\ \\
Case One: Assume that $x>0$. By Definition 1.17(ii) $(x)(x)>0$ and thus $x^{2}>0$. \\
Case Two: Assume that $x<0$. It follows from Proposition 1.18$(a)$ that $-x>0$. Then by Definition 1.17(ii) we have $(-x)(-x)>0$ but we know from Proposition 1.16$(d)$ that $(-x)(-x)=(x)(x)=x^{2}$ and thus we have $x^{2}>0$.\\ \\
Now to prove that $1>0$, note that $1 \neq 0$ so $1^{2}>0$ and thus $1>0$ since $1^{2}=1$ by the definition of $1$ in field axiom $(M4)$.\\
\item Proposition $1.18e$ asserts the following: If $0<x<y$, then $0<\dfrac{1}{y}<\dfrac{1}{x}$. \\ \\
We will first show that $\dfrac{1}{x}>0$. We know that $x>0$. Let $z \in F$ such that $z \leq 0$. Then $(-z)(x) \geq 0$ so $(z)(x) \leq 0$. By field axiom $(M5)$ and Proposition 1.18$(d)$, $(x) \bigg(\dfrac{1}{x} \bigg) = 1 > 0$ so it must be the case that $\dfrac{1}{x} >0$ by 1.17(ii). By a similar argument, namely replacing $x$ by $y$ in the above, we can verify that $\dfrac{1}{y}>0$. This tells us that \begin{equation}
0< \dfrac{1}{x}
\end{equation} and that \begin{equation}
0 < \dfrac{1}{y}
\end{equation} Thus, $0< \bigg( \dfrac{1}{x} \bigg) \bigg( \dfrac{1}{y} \bigg)$. By Proposition 1.18$(b)$, since $x<y$ by assumption, then we have $ \bigg( \dfrac{1}{x} \bigg) \bigg( \dfrac{1}{y} \bigg)x < \bigg( \dfrac{1}{x} \bigg) \bigg( \dfrac{1}{y} \bigg)y$ so $ \bigg( \dfrac{1}{y} \bigg)< \bigg( \dfrac{1}{x} \bigg)$. Using this with Eq.(1) and Eq.(2) shows 
\begin{equation}
0< \dfrac{1}{y} < \dfrac{1}{x}
\end{equation}
which is precisely what we wanted to prove. \\ \\
\end{enumerate}

\problem Theorem 1.20 \\

Prove the following theorem: \\ 
a) If $x \in R$, $y \in R$ and $x>0$ then there is a positive integer $n$ such that $nx>y$. \\
b) If $x \in R$, $y \in R$ and $x<y$ then there exists a $p \in \mathbb{Q}$ such that $x<p<y$. \\ \\
To prove statement (a) we will do so by contradiction. Assume $nx \leq y$, and let $S= \braces{ nx ~|~ n=1,2,3,4...}.$ Since we have a nonempty subset of $R$ which is bounded above by $y$, we may invoke the LUB property; that is, the supremum of $S$ exists. Let sup $S$ = $\alpha$. By the definition of the LUB, we have \begin{equation}
nx \leq \alpha
\end{equation} Since this statement holds for all $n$ in the set, it also holds for $n+1$. Thus we have \begin{equation}
(n+1)x \leq \alpha 
\end{equation}
\begin{equation*}
nx+x \leq \alpha
\end{equation*}
\begin{equation}
nx \leq \alpha - x
\end{equation} 
Recall that we assumed $x>0$ so $\alpha -x < \alpha$. By Equation (6) we have $nx \leq \alpha - x$ and this contradicts $\alpha$ being the LUB because we have found an even smaller upper bound for $S$. This proves (a). \\ \\
Armed with this tool, we will now use it to prove the second statement. Since $x<y$ this implies that $y-x>0$ by Definition 1.17(i). We will now take $y-x$ to be our $x$ in statement $(a)$ and we will let $1$ be our $y$ in statement $(a)$. Then there exists a positive integer $n$ such that $n(y-x)>1$. So we have
\begin{align*}
n(y-x)>1 \\
ny-nx>1
\end{align*} \begin{equation}
ny>nx+1
\end{equation}
Let us set Equation 7 aside and come back to it later. Let $m$ be the smallest integer such that $m>nx$. How do we know such an $m$ exists? Assume $m$ does not exist. Then that means that there is no smallest integer greater than some real number, $nx$. Let $\floor{nx}$  denote the integer part of $nx$. Then the successor of $\floor{nx}$ is $\floor{nx}+1$. Clearly, $\floor{nx}+1>nx$. So we have found an integer greater than $nx$ but how do we know this is the smallest such integer? Assume that there is even a smaller integer than $\floor{nx}+1$. Then this integer can be represented as $\floor{nx}+ \epsilon$. However, since all distinct integers differ by at least 1, $\epsilon$ must be exactly 1 otherwise $\floor{nx}+ \epsilon$ would not be an integer for $0< \epsilon <1$. Thus, such an $m$ exists and $m>nx$ and so \begin{equation}
\dfrac{m}{n} > x
\end{equation}. \\
Since $m$ is the smallest integer such that $m>nx$, then $m-1 \leq nx$. To see this, note if $m-1 > nx$, then we would have $m> m-1 > nx$ which contradicts our choice of $m$. So $m-1 \leq nx$ implies $m \leq nx+1$. So Equation 7 and $m \leq nx+1$ implies
\begin{equation}
m \leq nx+1 < ny 
\end{equation}
\begin{equation}
m < ny 
\end{equation}
\begin{equation}
\dfrac{m}{n} < y
\end{equation}
Putting Eq.8 and Eq.11 tells us
\begin{equation}
x < \dfrac{m}{n} < y 
\end{equation}
To complete the proof, we must show $\dfrac{m}{n}$ is rational. Since $m$ and $n$ are defined to be integers and $n$ cannot be zero since $n$ is positive, then $\dfrac{m}{n} \in \mathbb{Q}$.

\problem Chapter 1 Question 4 \\

Let $E$ be a nonempty subset of an ordered set; suppose $\alpha$ is a lower bound of $E$ and $\beta$ is an upper bound of $E$. Prove that $ \alpha \leq \beta$. \\ \\
Since $\alpha$ is a lower bound of $E$, we know that $\alpha \leq x$ for all $x \in E$. We are also given that $\beta$ is an upper bound of $E$ and so $x \leq \beta$ for all $x \in E$. By the transitivity of an ordered set, Definition 1.5(ii), we have $\alpha \leq x \leq \beta$ for all $x \in E$. From this we can conclude that  $ \alpha \leq \beta$.

\end{document}