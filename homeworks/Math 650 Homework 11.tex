\documentclass[12pt]{article} 
\usepackage{amsmath}
\usepackage{amsfonts}
\usepackage{amssymb}
\usepackage{color}
\usepackage{enumerate}
\usepackage[hyperfootnotes=false]{hyperref}

\newtheorem{theorem}{Theorem}[section]
\newtheorem{corollary}{Corollary}[theorem]
\newtheorem{lemma}[theorem]{Lemma}
\title{Math 650.2  Problem Set 11}
\author{Elliot Gangaram\\
\date{}
\ elliot.gangaram@gmail.com \\}
\include{preamble}


\newtoggle{spacingmode}
\begin{document}
\maketitle

\problem Prove that all $k-$cells are convex when the metric space is $\mathbb{R}^{k}$ with the usual distance function. \\ 

Recall that if $a_{i}<b_{i}$ for $i=1, \ldots, k$, the set of all points $\textbf{x}=(x_{1}, x_{2}, \ldots, x_{k})$ in $\mathbb{R}^{k}$ whose coordinates satisfy the inequalities $a_{i} \leq x_{i} \leq b_{i}$ $(1 \leq i \leq k)$ is called a $k$-cell. Intuitively, a $k-$cell in $\mathbb{R}^{1}$ is an interval, in $\mathbb{R}^{2}$ is a rectangle and $\mathbb{R}^{3}$ is a rectangular prism. Also recall that a set $E \subset \mathbb{R}^{k}$ is convex if 
\begin{equation}
\lambda \textbf{y} + (1-\lambda) \textbf{z} \in E
\end{equation}
whenever $\textbf{y} \in E, \textbf{z} \in E,$ and $0<\lambda<1$. We would like to show that all $k-$cells are convex when the metric space is $\mathbb{R}^{k}$ under the usual metric. \\ \\ 

Let $E$ be some $k-$cell and let $\textbf{y} \in E$ and $\textbf{z} \in E$. Consider $\lambda \textbf{y} + (1-\lambda) \textbf{z}$. We would like to show this vector belongs to $E$. For  simplicity call $(1-\lambda) = \beta$. So we have
\begin{align*}
& \lambda \textbf{y} + \beta \textbf{z} \\
& = \lambda (y_{1}, y_{2}, \ldots, y_{k})+ \beta (z_{1}, z_{2}, \ldots, z_{k}) \\
& = (\lambda y_{1}, \lambda y_{2}, \ldots, \lambda y_{k})+ (\beta z_{1}, \beta z_{2}, \ldots, \beta z_{k}) \\ 
& = (\lambda y_{1}+\beta z_{1} ,\lambda y_{2}+\beta z_{2} , \ldots, \lambda y_{k}+\beta z_{k} )
\end{align*}
If we can show that $a_{i} \leq \lambda y_{i}+\beta z_{i} \leq b_{i}$ $\forall i$, then we are done because  this tells us $\lambda \textbf{y} + (1-\lambda) \textbf{z} \in E$. Let us use our assumptions that $\textbf{y} \in E$ and $\textbf{z} \in E$. This implies that
\begin{align}
a_{i} \leq y_{i} \leq b_{i}
\end{align} and 
\begin{align}
a_{i} \leq z_{i} \leq b_{i}
\end{align}
Multiplying Eq. 2 by $\lambda$ yields
\begin{align}
\lambda a_{i} \leq \lambda y_{i} \leq \lambda  b_{i}
\end{align} and multiplying Eq. 3 by $\beta$ yields
\begin{align}
\beta a_{i} \leq \beta z_{i} \leq \beta b_{i}
\end{align}
Adding Equations 4 and 5 and simplifying yields
\begin{align}
& \lambda a_{i} + \beta a_{i} \leq \lambda y_{i} + \beta z_{i} \leq \lambda  b_{i} + \beta b_{i} \\ 
& = a_{i}(\lambda + \beta) \leq \lambda y_{i} + \beta z_{i}  \leq  b_{i}(\lambda + \beta)
\end{align}
But recall that $\lambda + \beta = \lambda + (1 - \lambda) = 1$ and so Equation 7 becomes
\begin{align}
a_{i} \leq \lambda y_{i}+\beta z_{i} \leq b_{i}
\end{align}
which is precisely what we needed to show.  Thus, $k-$cells are convex in $\mathbb{R}^{k}$. \\ \\

\problem Prove that all balls are convex when the metric space is $\mathbb{R}^{k}$ with the usual distance function. \\

Recall that the open (or closed) ball $B$ with center $\textbf{x}$ and radius $r$ is defined to be the set of all $y \in \mathbb{R}^{k}$ such that $|\textbf{y}-\textbf{x}|<r$ (or $|\textbf{y}-\textbf{x}| \leq r$) where $\textbf{x} \in \mathbb{R}^{k}$ and $r>0$. We would like to show that all balls are convex when the metric space is $\mathbb{R}^{k}$. Let $\textbf{y}$ and $\textbf{z}$ be two vectors inside the ball $B$ centered at $\textbf{x}$ of radius $r$. Observe that 
\begin{align*}
& \indent |\lambda\textbf{y} + (1-\lambda)\textbf{z}-\textbf{x}|\\
& = |\lambda(\textbf{y}-\textbf{x})+(1-\lambda)(\textbf{z}-\textbf{x})| \\ 
& \leq \lambda | \textbf{y}-\textbf{x}| + (1-\lambda)|\textbf{z}-\textbf{y}| \\
& < \lambda r + (1-\lambda)r \\
& = r
\end{align*}
where going from the third line to the fourth line follows because $|\textbf{y}-\textbf{x}|<r$ and $|\textbf{z}-\textbf{x}|<r$ since we assumed that $\textbf{y} \in B$ and $\textbf{z} \in B$.  Notice that this proof is for the open ball $B$. However, the same proof holds for the closed ball $B$ since the $<$ in the fourth line becomes $\leq$ which still leads us to our desired result. \\ \\ 

\problem Find a metric space where closed balls are \textsl{not} convex. \\ \\
In the previous question, we showed that $\mathbb{R}^{k}$ under the usual metric has the property that all open balls are convex. This leads us to a natural question - under what metric are balls not convex? Consider the metric space $\mathbb{R}^{2}$ along with the metric $d$ where $d$ is defined as follows: $d(\textbf{x}, \textbf{y})=d\big((x_{1},x_{2}),(y_{1},y_{2})\big) = \sqrt{|x_{1}-y_{1}|}+\sqrt{|x_{2}-y_{2}|}$. One must really check that $d$ is a metric. It is clear that $d$ is a real valued function and satisfies the first two properties of the definition of metric (see Rudin Definition 2.15). What is not so obvious is seeing that the function satisfies property $(c)$. To see this, observe that 
\begin{align*}
&   d(\textbf{x}, \textbf{y}) \\ 
& = d\big((x_{1},x_{2}),(y_{1},y_{2})\big)\\ 
& = \sqrt{|x_{1}-y_{1}|}+\sqrt{|x_{2}-y_{2}|} \\
& \leq  \sqrt{|x_{1}-z_{1}|}+\sqrt{|z_{1}-y_{1}|} + \sqrt{|x_{2}-z_{2}|}+\sqrt{|z_{2}-y_{2}|} \\ 
& = d(\textbf{x}, \textbf{z}) + d(\textbf{z}, \textbf{y})
\end{align*}  
which proves that $d$ is a metric on $\mathbb{R}^{2}$. Note that the inequality from line four follows from the fact{\footnote{For proof of this fact, observe that $a+b \leq a+b+2\sqrt{ab}$ and taking the square root of both sides yields $\sqrt{a+b} \leq \sqrt{a} + \sqrt{b}$.}} that $\sqrt{a+b} \leq \sqrt{a} + \sqrt{b}$ for $a, b \in \mathbb{R}^{+}$. \\ \\

Now consider the ball $B$ centered at $\mathbf{x}=\textbf{0}=(0,0)$ whose radius is 1. To show that this set is not convex, we must show there exists two vectors, $\textbf{u}$ and $\textbf{v}$ in the ball $B$ such that $\lambda \textbf{u} + (1- \lambda)\textbf{v} \notin B$. Let $\textbf{u} = (0.75,0)$ and let $\textbf{v}=(0,0.75)$. Note that these vectors are in $B$ since $d(\textbf{x}, \textbf{u})=d\big((0,0),(0.75,0)\big)=\sqrt{0.75}<1$ and similarly, $d(\textbf{x}, \textbf{v})=\sqrt{0.75}<1$. Now consider $\lambda \textbf{u} + (1- \lambda)\textbf{v} $. Take $\lambda=\dfrac{1}{2}$. Is is true that $\dfrac{1}{2} \textbf{u} + \dfrac{1}{2} \textbf{v}$ is in $B$? If it is, then $d(\textbf{x}, \dfrac{1}{2} \textbf{u} + \dfrac{1}{2} \textbf{v}) < 1$. Observe that 
\begin{align*}
   & d \big(\textbf{x}, \dfrac{1}{2} \textbf{u} + \dfrac{1}{2} \textbf{v} \big) \\ 
   & = d \big( (0,0), \dfrac{1}{2} (0.75,0) + \dfrac{1}{2} (0,0.75) \big) \\
   & = d \big( (0,0), (0.375, 0.375) \big) \\ 
   & = \sqrt{|0-0.375|}+\sqrt{|0-0.375|} \\
   & = \sqrt{0.375} + \sqrt{0.375} \\
   & \approx 1.22 > 1
\end{align*}
Thus the ball $B$ of radius 1 centered at the origin is not convex. \\ \\
 
\problem Illustrate an intuitive meaning of the terms defined on page 32. \\ \\

See attached. 
\end{document}